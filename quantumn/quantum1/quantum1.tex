\PassOptionsToPackage{unicode=true}{hyperref} % options for packages loaded elsewhere
\PassOptionsToPackage{hyphens}{url}
%
\documentclass[UTF8,twocolumn]{ctexart}
\usepackage{lmodern}
\usepackage{amssymb,amsmath}
\usepackage{ifxetex,ifluatex}
\usepackage{fixltx2e} % provides \textsubscript
\ifnum 0\ifxetex 1\fi\ifluatex 1\fi=0 % if pdftex
  \usepackage[T1]{fontenc}
  \usepackage[utf8]{inputenc}
  \usepackage{textcomp} % provides euro and other symbols
\else % if luatex or xelatex
  \usepackage{unicode-math}
  \defaultfontfeatures{Ligatures=TeX,Scale=MatchLowercase}
\fi
% use upquote if available, for straight quotes in verbatim environments
\IfFileExists{upquote.sty}{\usepackage{upquote}}{}
% use microtype if available
\IfFileExists{microtype.sty}{%
\usepackage[]{microtype}
\UseMicrotypeSet[protrusion]{basicmath} % disable protrusion for tt fonts
}{}
\IfFileExists{parskip.sty}{%
\usepackage{parskip}
}{% else
\setlength{\parindent}{0pt}
\setlength{\parskip}{6pt plus 2pt minus 1pt}
}
\usepackage{hyperref}
\hypersetup{
            pdfborder={0 0 0},
            breaklinks=true}
\urlstyle{same}  % don't use monospace font for urls
\setlength{\emergencystretch}{3em}  % prevent overfull lines
\providecommand{\tightlist}{%
  \setlength{\itemsep}{0pt}\setlength{\parskip}{0pt}}
\setcounter{secnumdepth}{0}
% Redefines (sub)paragraphs to behave more like sections
\ifx\paragraph\undefined\else
\let\oldparagraph\paragraph
\renewcommand{\paragraph}[1]{\oldparagraph{#1}\mbox{}}
\fi
\ifx\subparagraph\undefined\else
\let\oldsubparagraph\subparagraph
\renewcommand{\subparagraph}[1]{\oldsubparagraph{#1}\mbox{}}
\fi

% set default figure placement to htbp
\makeatletter
\def\fps@figure{htbp}
\makeatother


\date{}

\begin{document}

\hypertarget{ux7b2cux4e00ux7ae0-ux91cfux5b50ux529bux5b66ux7684ux5386ux53f2ux6e0aux6e90}{%
\section{第一章{ }量子力学的历史渊源}\label{ux7b2cux4e00ux7ae0-ux91cfux5b50ux529bux5b66ux7684ux5386ux53f2ux6e0aux6e90}}

\hypertarget{ux4e00ux91cfux5b50ux8bbaux7684ux5efaux7acb}{%
\subsection{一、量子论的建立}\label{ux4e00ux91cfux5b50ux8bbaux7684ux5efaux7acb}}

\hypertarget{ux7ecfux5178ux7269ux7406}{%
\subsubsection{1 经典物理}\label{ux7ecfux5178ux7269ux7406}}

\hypertarget{ux6210ux529f}{%
\paragraph{{ }1.1 成功}\label{ux6210ux529f}}

\begin{itemize}
\tightlist
\item
  牛顿运动定理:粒子性
\item
  麦克斯韦方程组:波动性
\end{itemize}

\hypertarget{ux56f0ux96be}{%
\paragraph{{ }1.2 困难}\label{ux56f0ux96be}}

\begin{itemize}
\tightlist
\item
  黑体辐射问题
\item
  光电效应
\item
  原子的线状光谱及其规律
\end{itemize}

\hypertarget{ux9002ux7528ux8303ux56f4}{%
\paragraph{{ }1.3 适用范围:}\label{ux9002ux7528ux8303ux56f4}}

 宏观、低速

\hypertarget{ux65e7ux91cfux5b50ux8bba}{%
\subsubsection{2 旧量子论}\label{ux65e7ux91cfux5b50ux8bba}}

\hypertarget{ux5efaux7acb}{%
\paragraph{{ }2.1 建立}\label{ux5efaux7acb}}

\begin{itemize}
\tightlist
\item
  19世纪末三大发现:x射线,放射性,电子
\item
  1900 普朗克 能量量子化:\(\varepsilon=h\upsilon\)
\item
  1905 爱因斯坦 光量子
\item
  1913 玻尔 量子态
\item
  1925 泡利 不相容原理
\item
  1925 乌仑别克、古兹米特 电子自旋假设
\end{itemize}

\hypertarget{ux6210ux529f-1}{%
\paragraph{{ }2.2 成功}\label{ux6210ux529f-1}}

\begin{itemize}
\tightlist
\item
  量子态概念获实验验证
\item
  提出角动量量子化
\item
  成功解释氢原子光谱
\end{itemize}

\hypertarget{ux5c40ux9650}{%
\paragraph{{ }2.3 局限}\label{ux5c40ux9650}}

\begin{itemize}
\tightlist
\item
  跃迁概念不清
\item
  无法解释氦原子光谱
\item
  无法解释原子如何组成分子
\end{itemize}

\hypertarget{ux65b0ux91cfux5b50ux8bba}{%
\subsubsection{3 新量子论}\label{ux65b0ux91cfux5b50ux8bba}}

\hypertarget{ux5efaux7acb-1}{%
\paragraph{{ }3.1 建立}\label{ux5efaux7acb-1}}

\begin{itemize}
\tightlist
\item
  1905 爱因斯坦 光子的波粒二象性
\item
  1917 康普顿 康普顿散射实验
\item
  1924 德布罗意 实物粒子的波粒二象性
\item
  1925-1928 玻恩、约旦 矩阵力学
\item
  1925-1928 薛定谔 波动力学
\item
  1925-1928 狄拉克 相对论量子力学
\item
  20世纪40年代 费曼 路径积分理论
\end{itemize}

\hypertarget{ux91cfux5b50ux529bux5b66}{%
\paragraph{{ }3.2 量子力学}\label{ux91cfux5b50ux529bux5b66}}

\begin{itemize}
\tightlist
\item
  定义:研究微观世界物质运动和变化的基本规律的科学
\item
  适用范围:微观 \& 宏观
\end{itemize}

\hypertarget{ux4e8cux666eux6717ux514b-ux7231ux56e0ux65afux5766ux7684ux5149ux91cfux5b50ux8bba}{%
\subsection{二、普朗克-爱因斯坦的光量子论}\label{ux4e8cux666eux6717ux514b-ux7231ux56e0ux65afux5766ux7684ux5149ux91cfux5b50ux8bba}}

\hypertarget{ux5149ux7684ux6ce2ux52a8ux6027}{%
\subsubsection{1 光的波动性}\label{ux5149ux7684ux6ce2ux52a8ux6027}}

 光的干涉、衍射、偏振,光的电磁理论

\hypertarget{ux9ed1ux4f53ux8f90ux5c04}{%
\subsubsection{2 黑体辐射}\label{ux9ed1ux4f53ux8f90ux5c04}}

\begin{itemize}
\tightlist
\item
  维恩公式:短波相符,长波不符
\item
  瑞利-金斯公式:长波相符,短波不符
\item
  普朗克公式:\\\(\varepsilon =h\upsilon\)\(\rho(\upsilon)d\upsilon=\frac{8\pi h\upsilon^3}{c^3}\frac{1}{e^{h\upsilon/kT}-1}d\upsilon\)
\end{itemize}

\hypertarget{ux5149ux7535ux6548ux5e94}{%
\subsubsection{3 光电效应}\label{ux5149ux7535ux6548ux5e94}}

\begin{itemize}
\tightlist
\item
  爱因斯坦光电方程:\(K_e=h\upsilon-W_0\)
\item
  光子的能量:\(E=h\upsilon=\hbar\omega\)
\item
  光子的动量:\(\vec{p}=\frac{h}{\lambda}\vec{n}=\hbar\vec{k}\)
\end{itemize}

\hypertarget{ux5eb7ux666eux987fux6548ux5e94}{%
\subsubsection{4 康普顿效应}\label{ux5eb7ux666eux987fux6548ux5e94}}

\begin{itemize}
\tightlist
\item
  现象:电磁波被散射后波长随散射角的增大而增大
\item
  解释:\(\Delta\lambda=\lambda'-\lambda=\frac{4\pi\hbar}{\mu_0c}sin^2\frac{\theta}{2}\)
\item
  意义:实验证实了光的波粒二象性
\end{itemize}

\hypertarget{ux4e09ux73bbux5c14ux7684ux539fux5b50ux7ed3ux6784ux6a21ux578b}{%
\subsection{三、玻尔的原子结构模型}\label{ux4e09ux73bbux5c14ux7684ux539fux5b50ux7ed3ux6784ux6a21ux578b}}

\hypertarget{ux6c22ux539fux5b50ux5149ux8c31}{%
\subsubsection{1 氢原子光谱}\label{ux6c22ux539fux5b50ux5149ux8c31}}

\hypertarget{ux5df4ux5c14ux672bux516cux5f0f}{%
\paragraph{{ }1.1 巴尔末公式}\label{ux5df4ux5c14ux672bux516cux5f0f}}

\begin{itemize}
\tightlist
\item
  巴尔末公式:\\\(\upsilon=R_Hc(\frac{1}{n'^2}-\frac{1}{n^2}),n'< n=1,2,3,...\)
\item
  里德伯常数:\(R_H=1.097\times 10^7m^{-1}\)
\end{itemize}

\hypertarget{ux7ecfux5178ux7406ux8bbaux7684ux56f0ux96be}{%
\paragraph{{ }1.2
经典理论的困难}\label{ux7ecfux5178ux7406ux8bbaux7684ux56f0ux96be}}

\begin{itemize}
\tightlist
\item
  无法解释原子的稳定性
\item
  无法解释原子光谱的分立性
\item
  无法解释并合原则
\end{itemize}

\hypertarget{ux5f17ux5170ux514b-ux8d6bux5179ux5b9eux9a8c}{%
\paragraph{{ }1.3
弗兰克-赫兹实验}\label{ux5f17ux5170ux514b-ux8d6bux5179ux5b9eux9a8c}}

\begin{itemize}
\tightlist
\item
  证明了原子能量的不连续性
\end{itemize}

\hypertarget{ux73bbux5c14ux6a21ux578b}{%
\subsubsection{2 玻尔模型}\label{ux73bbux5c14ux6a21ux578b}}

\hypertarget{ux57faux672cux5047ux8bbe}{%
\paragraph{{ }2.1 基本假设}\label{ux57faux672cux5047ux8bbe}}

\begin{itemize}
\tightlist
\item
  定态假设
\item
  跃迁假设:\(h\upsilon=|E_n-E_m|\)
\item
  量子化条件:电子的轨道角动量只能是\(\hbar\)的整数倍
\end{itemize}

\hypertarget{ux63a8ux5bfcux7ed3ux679c}{%
\paragraph{{ }2.2 推导结果}\label{ux63a8ux5bfcux7ed3ux679c}}

\begin{itemize}
\item
  氢原子能级:\(E_n=-\frac{\mu k_1^2e^4}{2\hbar^2}\frac{1}{n^2},n=1,2,3,...\)
\item
  里德伯常量:\(R_H=\frac{\mu k_1^2e^4}{4\pi\hbar^3c}\)
\end{itemize}

\hypertarget{ux63a8ux5e7fux7684ux91cfux5b50ux5316ux6761ux4ef6}{%
\subsubsection{3
推广的量子化条件}\label{ux63a8ux5e7fux7684ux91cfux5b50ux5316ux6761ux4ef6}}

\begin{itemize}
\tightlist
\item
  公式:\(\oint pdq=(n+\frac{1}{2})h,n=0,1,2,...\)
\item
  注:\(q\)为广义坐标,\(p\)为对应的广义动量,\(n\)为量子数
\end{itemize}

\hypertarget{ux73bbux5c14ux6a21ux578bux7684ux5c40ux9650ux6027ux53caux539fux56e0}{%
\subsubsection{4
玻尔模型的局限性及原因}\label{ux73bbux5c14ux6a21ux578bux7684ux5c40ux9650ux6027ux53caux539fux56e0}}

\hypertarget{ux5c40ux9650ux6027}{%
\paragraph{{ }4.1 局限性}\label{ux5c40ux9650ux6027}}

\begin{itemize}
\tightlist
\item
  不能给出谱线强度
\item
  不能解释复杂原子光谱
\item
  不能处理非束缚态问题
\item
  与经典理论不相容,物理本质不清楚
\end{itemize}

\hypertarget{ux539fux56e0}{%
\paragraph{{ }4.2 原因}\label{ux539fux56e0}}

 将微观粒子看作经典力学中的质点,将经典力学的规律用在微观粒子上

\hypertarget{ux56dbux5fb7ux5e03ux7f57ux610fux7684ux7269ux8d28ux6ce2ux5047ux8bf4}{%
\subsection{四、德布罗意的物质波假说}\label{ux56dbux5fb7ux5e03ux7f57ux610fux7684ux7269ux8d28ux6ce2ux5047ux8bf4}}

\hypertarget{ux5fb7ux5e03ux7f57ux610fux5047ux8bf4}{%
\subsubsection{1
德布罗意假说}\label{ux5fb7ux5e03ux7f57ux610fux5047ux8bf4}}

\hypertarget{ux5fb7ux5e03ux7f57ux610fux5047ux8bf4-1}{%
\paragraph{{ }1.1
德布罗意假说}\label{ux5fb7ux5e03ux7f57ux610fux5047ux8bf4-1}}

 微观粒子具有波粒二象性

\hypertarget{ux5fb7ux5e03ux7f57ux610fux5173ux7cfb}{%
\paragraph{{ }1.2
德布罗意关系}\label{ux5fb7ux5e03ux7f57ux610fux5173ux7cfb}}

\begin{itemize}
\tightlist
\item
  \(E=h\upsilon=\hbar\omega\)
\item
  \(\vec{p}=\frac{h}{\lambda}\vec{n}=\hbar\vec{k}\)
\end{itemize}

\hypertarget{ux5fb7ux5e03ux7f57ux610fux6ce2}{%
\subsubsection{2 德布罗意波}\label{ux5fb7ux5e03ux7f57ux610fux6ce2}}

 \(\Psi(\vec{r},t)=Ae^{i(\vec{k}\cdot\vec{r}-\omega t)}=Ae^{i(\vec{p}\cdot\vec{r}-Et)/\hbar}\)

\hypertarget{ux5b9eux9a8cux9a8cux8bc1}{%
\subsubsection{3 实验验证}\label{ux5b9eux9a8cux9a8cux8bc1}}

\begin{itemize}
\tightlist
\item
  实验:电子衍射实验
\item
  意义:证明了微观粒子的波粒二象性
\end{itemize}

\end{document}
