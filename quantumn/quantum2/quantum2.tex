\PassOptionsToPackage{unicode=true}{hyperref} % options for packages loaded elsewhere
\PassOptionsToPackage{hyphens}{url}
%
\documentclass[UTF8,twocolumn]{ctexart}
\usepackage{lmodern}
\usepackage{amssymb,amsmath}
\usepackage{ifxetex,ifluatex}
\usepackage{fixltx2e} % provides \textsubscript
\ifnum 0\ifxetex 1\fi\ifluatex 1\fi=0 % if pdftex
  \usepackage[T1]{fontenc}
  \usepackage[utf8]{inputenc}
  \usepackage{textcomp} % provides euro and other symbols
\else % if luatex or xelatex
  \usepackage{unicode-math}
  \defaultfontfeatures{Ligatures=TeX,Scale=MatchLowercase}
\fi
% use upquote if available, for straight quotes in verbatim environments
\IfFileExists{upquote.sty}{\usepackage{upquote}}{}
% use microtype if available
\IfFileExists{microtype.sty}{%
\usepackage[]{microtype}
\UseMicrotypeSet[protrusion]{basicmath} % disable protrusion for tt fonts
}{}
\IfFileExists{parskip.sty}{%
\usepackage{parskip}
}{% else
\setlength{\parindent}{0pt}
\setlength{\parskip}{6pt plus 2pt minus 1pt}
}
\usepackage{hyperref}
\hypersetup{
            pdfborder={0 0 0},
            breaklinks=true}
\urlstyle{same}  % don't use monospace font for urls
\setlength{\emergencystretch}{3em}  % prevent overfull lines
\providecommand{\tightlist}{%
  \setlength{\itemsep}{0pt}\setlength{\parskip}{0pt}}
\setcounter{secnumdepth}{0}
% Redefines (sub)paragraphs to behave more like sections
\ifx\paragraph\undefined\else
\let\oldparagraph\paragraph
\renewcommand{\paragraph}[1]{\oldparagraph{#1}\mbox{}}
\fi
\ifx\subparagraph\undefined\else
\let\oldsubparagraph\subparagraph
\renewcommand{\subparagraph}[1]{\oldsubparagraph{#1}\mbox{}}
\fi

% set default figure placement to htbp
\makeatletter
\def\fps@figure{htbp}
\makeatother

\date{}

\begin{document}

\hypertarget{ux7b2cux4e8cux7ae0ux6ce2ux51fdux6570ux4e0eux859bux5b9aux8c14ux65b9ux7a0b}{%
\section{第二章{ }波函数与薛定谔方程}\label{ux7b2cux4e8cux7ae0ux6ce2ux51fdux6570ux4e0eux859bux5b9aux8c14ux65b9ux7a0b}}

\hypertarget{ux4e00ux6ce2ux51fdux6570ux53caux5176ux7edfux8ba1ux8be0ux91ca}{%
\subsection{一、波函数及其统计诠释}\label{ux4e00ux6ce2ux51fdux6570ux53caux5176ux7edfux8ba1ux8be0ux91ca}}

\hypertarget{ux72b6ux6001ux7684ux63cfux8ff0}{%
\subsubsection{1 状态的描述}\label{ux72b6ux6001ux7684ux63cfux8ff0}}

\hypertarget{ux7ecfux5178ux529bux5b66ux8d28ux70b9}{%
\paragraph{{ }1.1
经典力学:质点}\label{ux7ecfux5178ux529bux5b66ux8d28ux70b9}}

\begin{itemize}
\tightlist
\item
  每一时刻具有确定的\(\vec{r},\,\vec{p}(v)\)
\item
  其他力学量可表示为\(\vec{r},\,\vec{p}\)的函数
\item
  其状态变化服从牛顿运动定律
\end{itemize}

\hypertarget{ux91cfux5b50ux529bux5b66ux6ce2ux51fdux6570}{%
\paragraph{{ }1.2
量子力学:波函数}\label{ux91cfux5b50ux529bux5b66ux6ce2ux51fdux6570}}

\begin{itemize}
\tightlist
\item
  粒子不可能同时具有确定的\(\vec{r}\)和\(\vec{p}\)
\item
  粒子的状态由波函数\(\Psi(\vec{r},t)\)描述
\item
  其状态变化服从薛定谔方程
\end{itemize}

\hypertarget{ux6ce2ux7c92ux4e8cux8c61ux6027}{%
\subsubsection{2 波粒二象性}\label{ux6ce2ux7c92ux4e8cux8c61ux6027}}

\hypertarget{ux5b9eux7269ux7c92ux5b50ux7684ux6ce2ux7c92ux4e8cux8c61ux6027}{%
\paragraph{{ }2.1
实物粒子的波粒二象性}\label{ux5b9eux7269ux7c92ux5b50ux7684ux6ce2ux7c92ux4e8cux8c61ux6027}}

\begin{itemize}
\tightlist
\item
  理论:\(\lambda=\frac{h}{p},\,\upsilon=\frac{E}{h}\)
\item
  实验:\(C_{60}\)分子的干涉实验
\end{itemize}

\hypertarget{ux5b9eux7269ux7c92ux5b50ux4e8cux8c61ux6027ux7684ux7406ux89e3}{%
\paragraph{{ }2.2
实物粒子二象性的理解}\label{ux5b9eux7269ux7c92ux5b50ux4e8cux8c61ux6027ux7684ux7406ux89e3}}

\begin{itemize}
\tightlist
\item
  粒子性:与物质相互作用时的``整体性'',具有集中的能量\(E\)和动量\(\vec{p}\)
\item
  波动性:在空间传播时的``可叠加性'',具有波长\(\lambda\)和波矢\(\vec{k}\)
\item
  物质波包观点:夸大了波动性,抹杀了粒子性
\item
  疎密波观点:夸大了粒子性,抹杀了波动性
\end{itemize}

\hypertarget{ux6ce2ux51fdux6570ux7684ux7edfux8ba1ux8be0ux91ca}{%
\subsubsection{3
波函数的统计诠释}\label{ux6ce2ux51fdux6570ux7684ux7edfux8ba1ux8be0ux91ca}}

\hypertarget{ux6982ux7387ux6ce2}{%
\paragraph{{ }3.1 概率波}\label{ux6982ux7387ux6ce2}}

\begin{itemize}
\tightlist
\item
  粒子的状态可由波函数\(\Psi(\vec{r},t)\)完全描述
\item
  其模平方\(|\Psi(\vec{r},t)|^2\)表示粒子空间分布的概率密度,波函数本身称为概率振幅
\end{itemize}

\hypertarget{ux6ce2ux51fdux6570ux7684ux5f52ux4e00}{%
\paragraph{{ }3.2
波函数的归一}\label{ux6ce2ux51fdux6570ux7684ux5f52ux4e00}}

\begin{itemize}
\tightlist
\item
  空间概率密度:\(w(\vec{r},t)=|\Psi(\vec{r},t)|^2\)
\item
  归一化条件:
  \begin{center}
    \(\int_\infty w(\vec{r},t)d\tau=\int_\infty|\Psi(\vec{r},t)|^2d\tau=1\)
  \end{center}
\item
  注1:即使归一化后,波函数仍有一整体位相因子\(e^{i\delta}\)不能确定
\item
  注2:某些理想(非物理)情况无法归一,如平面波
\end{itemize}

\hypertarget{ux591aux7c92ux5b50ux6ce2ux51fdux6570}{%
\paragraph{{ }3.3
多粒子波函数}\label{ux591aux7c92ux5b50ux6ce2ux51fdux6570}}

\begin{itemize}
\tightlist
\item
  波函数:\(\Psi(\vec{r}_1,\vec{r}_2,\ldots,\vec{r}_N;t)\)
\item
  归一化条件:
  \begin{center}
    \(\int_\infty^{(N)}\Psi(\vec{r}_1,\vec{r}_2,\ldots,\vec{r}_N;t)d^3\vec{r}_1d^3\vec{r}_2\ldots d^3\vec{r}_N=1\)
  \end{center}
\end{itemize}

\hypertarget{ux7edfux8ba1ux8be0ux91caux5bf9ux6ce2ux51fdux6570ux7684ux8981ux6c42}{%
\paragraph{{ }3.4
统计诠释对波函数的要求}\label{ux7edfux8ba1ux8be0ux91caux5bf9ux6ce2ux51fdux6570ux7684ux8981ux6c42}}

\begin{itemize}
\tightlist
\item
  波函数平方可积
\item
  满足归一化条件,但不排除某些理想波函数
\item
  \(|\Psi(\vec{r})|^2\)单值,但不要求\(\Psi(\vec{r})\)单值
\item
  一般\(\Psi\)和\(\nabla\Psi\)连续,但\(\nabla\Psi\)在势能无限大跳变处可以不连续
\end{itemize}

\hypertarget{ux4e8cux91cfux5b50ux6001ux53e0ux52a0ux539fux7406}{%
\subsection{二、量子态叠加原理}\label{ux4e8cux91cfux5b50ux6001ux53e0ux52a0ux539fux7406}}

\hypertarget{ux91cfux5b50ux6001ux53caux5176ux8868ux8c61}{%
\subsubsection{1
量子态及其表象}\label{ux91cfux5b50ux6001ux53caux5176ux8868ux8c61}}

\hypertarget{ux6001ux51fdux6570}{%
\paragraph{{ }1.1 态函数}\label{ux6001ux51fdux6570}}

\begin{itemize}
\tightlist
\item
  动量表象:\(c(\vec{p})=\frac{1}{(2\pi\hbar)^{3/2}}\int\psi(\vec{r})e^{-i\vec{p}\cdot\vec{r}/\hbar}d^3\vec{r}\)
\item
  坐标表象:\(\psi(\vec{r})=\frac{1}{(2\pi\hbar)^{3/2}}\int c(\vec{p})e^{i\vec{p}\cdot\vec{r}/\hbar}d^3\vec{p}\)
\item
  注:\(\psi(\vec{r})\)和\(c(\vec{p})\)构成傅里叶变换对
\end{itemize}

\hypertarget{ux8868ux8c61}{%
\paragraph{{ }1.2 表象}\label{ux8868ux8c61}}

\begin{itemize}
\tightlist
\item
  含义:量子力学中态和力学量的表示方式
\item
  常用表象:坐标表象\(\psi(\vec{r})\),动量表象\(c(\vec{p})\)等
\end{itemize}

\hypertarget{ux91cfux5b50ux6001ux53e0ux52a0ux539fux7406}{%
\subsubsection{2
量子态叠加原理}\label{ux91cfux5b50ux6001ux53e0ux52a0ux539fux7406}}

\hypertarget{ux539fux7406}{%
\paragraph{{ }2.1 原理}\label{ux539fux7406}}

  若\(\Psi_1,\Psi_2,\ldots,\Psi_n\)是体系可能状态,则\(\Phi=\sum_nc_n\Psi_n\)也是体系的可能状态,其中\(c_n\)为复常数

\hypertarget{ux8ba8ux8bba}{%
\paragraph{{ }2.2 讨论}\label{ux8ba8ux8bba}}

\begin{itemize}
\tightlist
\item
  态矢量集合\{\(\Psi\)\}对线性叠加封闭,故构成一个线性空间
\item
  态叠加原理要求态矢量的时间演化方程为线性齐次方程
\item
  叠加态\(\Phi=\sum_nc_n\Psi_n\)处于态\(\Psi_k\)的概率与\(|c_k|^2\)成正比
\end{itemize}

\hypertarget{ux4e0eux7ecfux5178ux53e0ux52a0ux539fux7406ux7684ux533aux522b}{%
\paragraph{{ }2.3
与经典叠加原理的区别}\label{ux4e0eux7ecfux5178ux53e0ux52a0ux539fux7406ux7684ux533aux522b}}

\begin{itemize}
\tightlist
\item
  \(\Psi\)和\(c\Psi\)描述的是同一个状态
\item
  波函数无直接物理意义,其叠加为概率波函数的叠加,而非概率密度的叠加
\end{itemize}

\hypertarget{ux4e09ux859bux5b9aux8c14ux65b9ux7a0b}{%
\subsection{三、薛定谔方程}\label{ux4e09ux859bux5b9aux8c14ux65b9ux7a0b}}

\hypertarget{ux5efaux7acb}{%
\subsubsection{1 建立}\label{ux5efaux7acb}}

\hypertarget{ux7b97ux7b26}{%
\paragraph{{ }1.1 算符}\label{ux7b97ux7b26}}

\begin{itemize}
\tightlist
\item
  能量算符:\(E\to i\hbar\frac{\partial}{\partial t}\)
\item
  动量算符:\(p\to -i\hbar\nabla\)
\end{itemize}

\hypertarget{ux65b9ux7a0b}{%
\paragraph{{ }1.2 方程}\label{ux65b9ux7a0b}}

\begin{itemize}
\tightlist
\item
  能量守恒:\(E=\frac{\vec{p}^2}{2m}+U(\vec{r},t)\)
\item
  薛定谔方程:\(i\hbar\frac{\partial\Psi}{\partial t}=-\frac{\hbar^2}{2m}\nabla^2\Psi+U(\vec{r},t)\Psi\)
\end{itemize}

\hypertarget{ux8ba8ux8bba-1}{%
\subsubsection{2 讨论}\label{ux8ba8ux8bba-1}}

\hypertarget{ux8fdeux7eedux6027ux65b9ux7a0b}{%
\paragraph{{ }2.1 连续性方程}\label{ux8fdeux7eedux6027ux65b9ux7a0b}}

\begin{itemize}
\tightlist
\item
  空间概率密度:\(w(\vec{r},t)=|\Psi(\vec{r},t)|^2=\Psi^*(\vec{r},t)\Psi(\vec{r},t)\)
\item
  概率流密度矢量:\(\vec{J}=\frac{i\hbar}{2m}(\Psi\nabla\Psi^* -\Psi^*\nabla\Psi)\)
\item
  连续性方程:\(\frac{\partial w}{\partial t}+\nabla\cdot\vec{J}=0\)
\item
  物理意义:概率守恒,单位时间内体积V中增加的概率,等于从边界面流入V的概率通量
\end{itemize}

\hypertarget{ux8d28ux91cfux5b88ux6052ux5b9aux5f8bux4e0eux7535ux8377ux5b88ux6052ux5b9aux5f8b}{%
\paragraph{{ }2.2
质量守恒定律与电荷守恒定律}\label{ux8d28ux91cfux5b88ux6052ux5b9aux5f8bux4e0eux7535ux8377ux5b88ux6052ux5b9aux5f8b}}

\begin{itemize}
\tightlist
\item
  质量守恒:\(\frac{\partial w_m}{\partial t}+\nabla\cdot\vec{J_m}=0\)
\item
  电荷守恒:\(\frac{\partial w_q}{\partial t}+\nabla\cdot\vec{J_q}=0\)
\end{itemize}

\hypertarget{ux8bf4ux660e}{%
\paragraph{{ }2.3 说明}\label{ux8bf4ux660e}}

\begin{itemize}
\tightlist
\item
  薛定谔方程是量子力学的一个基本假设
\item
  薛定谔方程是线性偏微分方程,满足态叠加原理
\end{itemize}

\hypertarget{ux5b9aux6001ux859bux5b9aux8c14ux65b9ux7a0b}{%
\subsubsection{3
定态薛定谔方程}\label{ux5b9aux6001ux859bux5b9aux8c14ux65b9ux7a0b}}

\hypertarget{ux5b9aux6001ux859bux5b9aux8c14ux65b9ux7a0b-1}{%
\paragraph{{ }3.1
定态薛定谔方程}\label{ux5b9aux6001ux859bux5b9aux8c14ux65b9ux7a0b-1}}

\begin{itemize}
\tightlist
\item
  定义:粒子的势能函数\(U\)与时间\(t\)无关的稳定势场问题
\item
  定态波函数:\(\Psi(\vec{r},t)=e^{-\frac{iE}{\hbar}t}\psi(\vec{r})\)
\item
  定态薛定谔方程:\(-\frac{\hbar^2}{2m}\nabla^2\psi+U(\vec{r})\psi=E\psi\)
\end{itemize}

\hypertarget{ux5b9aux6001ux4e0eux975eux5b9aux6001}{%
\paragraph{{ }3.2
定态与非定态}\label{ux5b9aux6001ux4e0eux975eux5b9aux6001}}

\begin{itemize}
\tightlist
\item
  定态:体系的能量有确定值的状态
\item
  非定态:由若干个能量不同的本征态叠加所形成的的态,即\(\Psi(\vec{r},t)=\sum_Ec_E\psi(\vec{r})e^{-\frac{iE}{\hbar}t}\)
\end{itemize}

\hypertarget{ux5b9aux6001ux7684ux7279ux5f81}{%
\paragraph{{ }3.3 定态的特征}\label{ux5b9aux6001ux7684ux7279ux5f81}}

\begin{itemize}
\tightlist
\item
  粒子的空间概率密度\(\omega(\vec{r})\)和概率流密度\(\vec{J}\)不随时间改变
\item
  任何不显含时力学量的平均值不随时间改变
\item
  任何不显含时力学量的测值概率分布不随时间改变
\end{itemize}

\hypertarget{ux54c8ux5bc6ux987fux7b97ux7b26}{%
\subsubsection{4 哈密顿算符}\label{ux54c8ux5bc6ux987fux7b97ux7b26}}

\begin{itemize}
\tightlist
\item
  薛定谔方程的普遍表达:\(i\hbar\frac{\partial\Psi}{\partial t}=\hat{H}\Psi\)
\item
  单粒子的哈密顿算符:\(\hat{H}=-\frac{\hbar^2}{2m}\nabla^2+U(\vec{r})\)
\item
  多粒子的汉密顿算符:\(\hat{H}=\sum_{i=1}^N(-\frac{\hbar^2}{2m_i}\nabla_i^2+U_i(\vec{r_i}))+V(\vec{r_1},\vec{r_2},\ldots,\vec{r_N})\)
\item
  能量本征方程:\(\hat{H}\Psi=E\Psi\)
\end{itemize}

\hypertarget{ux56dbux4e00ux7ef4ux8fd0ux52a8ux95eeux9898ux7684ux4e00ux822cux5206ux6790}{%
\subsection{四、一维运动问题的一般分析}\label{ux56dbux4e00ux7ef4ux8fd0ux52a8ux95eeux9898ux7684ux4e00ux822cux5206ux6790}}

\hypertarget{ux4e00ux7ef4ux5b9aux6001ux859bux5b9aux8c14ux65b9ux7a0b}{%
\subsubsection{1
一维定态薛定谔方程}\label{ux4e00ux7ef4ux5b9aux6001ux859bux5b9aux8c14ux65b9ux7a0b}}

\begin{itemize}
\tightlist
\item
  方程:\([-\frac{\hbar}{2m}\frac{\partial^2}{\partial x^2}+U(x)]\psi(x)=E\psi(x)\)
\item
  其中\(E,\,U(x)\)均为实数
\end{itemize}

\hypertarget{ux4e00ux7ef4ux5b9aux6001ux7684ux5206ux7c7b}{%
\subsubsection{2
一维定态的分类}\label{ux4e00ux7ef4ux5b9aux6001ux7684ux5206ux7c7b}}

\hypertarget{ux675fux7f1aux6001ux4e0eux975eux675fux7f1aux6001}{%
\paragraph{{ }2.1
束缚态与非束缚态}\label{ux675fux7f1aux6001ux4e0eux975eux675fux7f1aux6001}}

\begin{itemize}
\tightlist
\item
  束缚态:粒子局限在有限的空间中,即\(lim_{x\to\pm\infty}\psi(x)=0\)
\item
  非束缚态:粒子可以出现在无限远处的状态,即\(lim_{x\to+\infty}\psi(x)>0\)或\(lim_{x\to-\infty}\psi(x)>0\)
\end{itemize}

\hypertarget{ux7b80ux5e76ux4e0eux975eux7b80ux5e76}{%
\paragraph{{ }2.2
简并与非简并}\label{ux7b80ux5e76ux4e0eux975eux7b80ux5e76}}

\begin{itemize}
\tightlist
\item
  定义:若对于给定的能级\(E\),只有一个线性无关的波函数存在,则称该能级是非简并的;否则称其为简并的
\item
  简并度:简并态中线性独立的波函数个数称为其简并度
\end{itemize}

\hypertarget{ux4e00ux7ef4ux5b9aux6001ux859bux5b9aux8c14ux65b9ux7a0bux7684ux6027ux8d28}{%
\subsubsection{3
一维定态薛定谔方程的性质}\label{ux4e00ux7ef4ux5b9aux6001ux859bux5b9aux8c14ux65b9ux7a0bux7684ux6027ux8d28}}

\hypertarget{ux5b9aux74061-ux5171ux8f6dux5b9aux7406}{%
\paragraph{{ }定理1
共轭定理}\label{ux5b9aux74061-ux5171ux8f6dux5b9aux7406}}

\begin{itemize}
\tightlist
\item
  若\(\psi(x)\)是定态方程的解,则\(\psi(x)^*\)也是方程的解,且能量相同
\item
  推论:若某能量本征值\(E\)的解无简并,则可取为实解
\end{itemize}

\hypertarget{ux5b9aux74062}{%
\paragraph{{ }定理2}\label{ux5b9aux74062}}

  对于任意能量本征值\(E\),总可以找到定态方程的一组实解,其线性组合可以表示属于\(E\)的任何解

\hypertarget{ux5b9aux74063-ux53cdux5c04ux5b9aux7406}{%
\paragraph{{ }定理3
反射定理}\label{ux5b9aux74063-ux53cdux5c04ux5b9aux7406}}

 3.1 定理

  若\(U(x)\)具有空间反射不变性,即\(U(x)=U(-x)\),那么若\(\psi(x)\)是方程的解,则\(\psi(-x)\)也是方程的解,且能量相同

 3.2 宇称

\begin{itemize}
\tightlist
\item
  空间反射算符:\(\hat{P}\Psi(x)=\Psi(-x)\)
\item
  本征方程:\(\hat{P}\Psi(x)=\pi\Psi(x)\)
\item
  宇称:空间反射算符的本征值,\(\pi=1\)为偶宇称,\(\pi=-1\)为奇宇称
\end{itemize}

 3.3 推论

  若\(U(x)=U(-x)\),且某能量本征值\(E\)的解无简并,则该解必有确定的宇称

\hypertarget{ux5b9aux74064}{%
\paragraph{{ }定理4}\label{ux5b9aux74064}}

  若\(U(x)=U(-x)\),则对于任意能量本征值\(E\),总可以找到一组有确定宇称的解,其线性组合可以表示属于\(E\)的任何解

\hypertarget{ux5b9aux74065}{%
\paragraph{{ }定理5}\label{ux5b9aux74065}}

  若\(U(x)\)的不连续点跳变值有限,则能量本征函数\(\psi(x)\)及其导数\(\psi(x)'\)连续

\hypertarget{ux5b9aux74066-wronskianux5b9aux7406}{%
\paragraph{{ }定理6
Wronskian定理}\label{ux5b9aux74066-wronskianux5b9aux7406}}

\begin{itemize}
\tightlist
\item
  Wronskian行列式:
  \begin{center}
    \(\psi_1(x)'\psi_2(x)-\psi_2(x)'\psi_1(x)\)
  \end{center}
  称为\(\psi_1(x),\,\psi_2(x)\)的Wronskian行列式
\item
  若\(\psi_1(x)\)和\(\psi_2(x)\)均为方程的解且能量相同,则\(\psi_1(x),\,\psi_2(x)\)的Wronskian行列式为与x无关的常数,即\(\psi_1(x)'\psi_2(x)-\psi_2(x)'\psi_1(x)=c\)
\end{itemize}

\hypertarget{ux5b9aux74067-ux4e0dux7b80ux5e76ux5b9aux7406}{%
\paragraph{{ }定理7
不简并定理}\label{ux5b9aux74067-ux4e0dux7b80ux5e76ux5b9aux7406}}

\begin{itemize}
\tightlist
\item
  定理:设\(U(x)\)为规则势场(即无奇点),若粒子存在束缚态,则该状态一定是非简并的
\item
  注:对于常见的非规则势场(如无限深势阱,\(\delta\)势阱),上述定理仍然成立
\end{itemize}

\hypertarget{ux4e94ux4e00ux7ef4ux65e0ux9650ux6df1ux52bfux9631ux548cux65b9ux52bfux9631}{%
\subsection{五、一维无限深势阱和方势阱}\label{ux4e94ux4e00ux7ef4ux65e0ux9650ux6df1ux52bfux9631ux548cux65b9ux52bfux9631}}

\hypertarget{ux4e00ux7ef4ux65e0ux9650ux6df1ux52bfux9631}{%
\subsubsection{1
一维无限深势阱}\label{ux4e00ux7ef4ux65e0ux9650ux6df1ux52bfux9631}}

\hypertarget{ux52bfux80fdux51fdux6570}{%
\paragraph{{ }1.1 势能函数}\label{ux52bfux80fdux51fdux6570}}

\[U(x)=\begin{cases} 
        0, & 0<x<a\\
        \infty, & x<0,x>a
\end{cases}\]

\hypertarget{ux5b9aux6001ux859bux5b9aux8c14ux65b9ux7a0b-2}{%
\paragraph{{ }1.2
定态薛定谔方程}\label{ux5b9aux6001ux859bux5b9aux8c14ux65b9ux7a0b-2}}

\begin{itemize}
\tightlist
\item
  阱内:\(-\frac{\hbar^2}{2m}\frac{d^2}{dx^2}\psi(x)=E\psi(x)\)
\item
  阱外:\((-\frac{\hbar^2}{2m}\frac{d^2}{dx^2}+\infty)\psi(x)=E\psi(x)\)
\end{itemize}

\hypertarget{ux5217ux5199ux901aux89e3}{%
\paragraph{{ }1.3 列写通解}\label{ux5217ux5199ux901aux89e3}}

\begin{itemize}
\tightlist
\item
  阱内:\(\psi(x)=Asin(kx+\delta)\),其中\(k=\sqrt{\frac{2mE}{\hbar^2}}\)
\item
  阱外:\(\psi(x)=0\)
\end{itemize}

\hypertarget{ux8fb9ux754cux6761ux4ef6}{%
\paragraph{{ }1.4 边界条件}\label{ux8fb9ux754cux6761ux4ef6}}

\begin{itemize}
\tightlist
\item
  利用边界条件\(\psi(0)=0,\,\psi(x)=0\),得\\\(ka=n\pi,\,n=1,2,3,\ldots\)
\item
  能级:\(E_n=\frac{\hbar^2\pi^2n^2}{2ma^2},\,n=1,2,3,\ldots\)
\item
  波函数:\(\psi_n(x)=A_nsin(\frac{n\pi}{a}x),\,n=1,2,3,\ldots\)
\end{itemize}

\hypertarget{ux5f52ux4e00ux5316ux6761ux4ef6}{%
\paragraph{{ }1.5 归一化条件}\label{ux5f52ux4e00ux5316ux6761ux4ef6}}

\begin{itemize}
\tightlist
\item
  利用归一化条件\(\int_0^a|\psi_n(x)|^2dx=1\),得\\\(A_n=\sqrt{\frac{2}{a}}\)
\item
  归一化波函数: \[\Psi_n(x,t)=\begin{cases} 
        \sqrt{\frac{2}{a}}sin(\frac{n\pi}{a}x)e^{-\frac{iE_n}{\hbar}t}, & 0<x<a\\
        0, & x<0,x>a
  \end{cases}\]
  \begin{center}
    \(n=1,2,3,\ldots\)
  \end{center}
\end{itemize}

\hypertarget{ux8ba8ux8bba-2}{%
\paragraph{{ }1.6 讨论}\label{ux8ba8ux8bba-2}}

\begin{itemize}
\tightlist
\item
  能量本征值:无限深势阱的能量是量子化的,其最低能级为\(E_1=\frac{\hbar^2\pi^2}{2ma^2}\)
\item
  本征函数系:本征函数是两两正交的,即\(\int_0^a\psi_m^*(x)\psi_n(x)dx=\delta_{m,n}\)
\end{itemize}

\hypertarget{ux6709ux9650ux6df1ux5bf9ux79f0ux65b9ux52bfux9631}{%
\subsubsection{2
有限深对称方势阱}\label{ux6709ux9650ux6df1ux5bf9ux79f0ux65b9ux52bfux9631}}

\hypertarget{ux52bfux80fdux51fdux6570-1}{%
\paragraph{{ }2.1 势能函数}\label{ux52bfux80fdux51fdux6570-1}}

\[U(x)=\begin{cases} 
        0, & |x|<\frac{a}{2}\\
        U_0, & |x|>\frac{a}{2}
\end{cases}\]

\hypertarget{ux5b9aux6001ux859bux5b9aux8c14ux65b9ux7a0b-3}{%
\paragraph{{ }2.2
定态薛定谔方程}\label{ux5b9aux6001ux859bux5b9aux8c14ux65b9ux7a0b-3}}

  考虑束缚态,即\(0<E<U_0\)时 

\begin{itemize}
  \tightlist
  \item
    阱内:
    \begin{center}
      \(\frac{d^2\psi(x)}{dx^2}+k^2\psi(x)=0\),其中\(k=\sqrt{\frac{2mE}{\hbar^2}}\)
    \end{center}
  \item
    阱外:
    \begin{center}
      \(\frac{d^2\psi(x)}{dx^2}-\beta^2\psi(x)=0\),其中\(\beta=\sqrt{\frac{2m(U_0-E)}{\hbar^2}}\)
    \end{center}
\end{itemize}

\hypertarget{ux5217ux5199ux901aux89e3-1}{%
\paragraph{{ }2.3 列写通解}\label{ux5217ux5199ux901aux89e3-1}}

\begin{itemize}
\tightlist
\item
  阱内:\(\psi(x)=Acos(kx)+Bsin(kx)\)
\item
  阱外:\(\psi(x)=Ce^{\beta x}+De^{-\beta x}\)
\end{itemize}

\hypertarget{ux7269ux7406ux5206ux6790}{%
\paragraph{{ }2.4 物理分析}\label{ux7269ux7406ux5206ux6790}}

\begin{itemize}
\tightlist
\item
  束缚态条件:注意到\(E<U_0\)为束缚态,有\(lim_{x\to\pm\infty}\psi(x)=0\),故波函数为
  \[\psi(x)=\begin{cases} 
        Ce^{\beta x}, & x<-\frac{a}{2}\\
        Acos(kx)+Bsin(kx), & -\frac{a}{2}<x<\frac{a}{2}\\
        De^{-\beta x}, & x>-\frac{a}{2}
  \end{cases}\]
\item
  势能对称性:注意到\(U(x)=U(-x)\),故定态波函数必有确定的宇称,下分偶宇称和奇宇称分别进行讨论
\end{itemize}

\hypertarget{ux8fb9ux754cux6761ux4ef6-1}{%
\paragraph{{ }2.5 边界条件}\label{ux8fb9ux754cux6761ux4ef6-1}}

\begin{itemize}
\tightlist
\item
  偶宇称:\(B=0,\,C=D\),由边界条件得 \[\begin{cases} 
        Acos(\frac{ka}{2})=De^{-\frac{\beta a}{2}}\\
        -kAsin(\frac{ka}{2})=-\beta De^{-\frac{\beta a}{2}}
  \end{cases}\Rightarrow k\,tan(\frac{ka}{2})=\beta\]
\item
  奇宇称:\(A=0,\,C=-D\),由边界条件得 \[\begin{cases} 
        Bsin(\frac{ka}{2})=De^{-\frac{\beta a}{2}}\\
        kBsin(\frac{ka}{2})=-\beta De^{-\frac{\beta a}{2}}
  \end{cases}\Rightarrow k\,cot(\frac{ka}{2})=-\beta\]
\item
  图解法:引入\(\xi=\frac{ka}{2},\,\eta=\frac{\beta a}{2}\),采用图解法解如下超越方程
  \[\begin{cases} 
        \eta=\xi\,tan\xi\\
        \eta^2+\xi^2=\frac{mU_0a^2}{2\hbar^2}
  \end{cases}or\,
  \begin{cases} 
        \eta=-\xi\,cot\xi\\
        \eta^2+\xi^2=\frac{mU_0a^2}{2\hbar^2}
  \end{cases}\]
\item
  能级:\(E_n=\frac{2\hbar^2}{ma^2}\xi_n^2\)
\end{itemize}

\hypertarget{ux8ba8ux8bba-3}{%
\paragraph{{ }2.6 讨论}\label{ux8ba8ux8bba-3}}

  由图解法可得:

\begin{itemize}
\tightlist
\item
  无论势阱深浅,至少存在一个束缚态(基态)
\item 
  能级的宇称奇偶相间,最低能级为偶宇称
\item 
  有限深势阱的各能级都低于无限深势阱能级;当\(n\to\infty\)时,各能级趋近于无限深势阱的响应能级
\item 
  束缚态能级总数\(N=1+[\frac{a}{\hbar\pi}\sqrt{2mU_0}\,]\)
\end{itemize}

\hypertarget{ux675fux7f1aux6001ux4e0eux79bbux6563ux8c31}{%
\subsubsection{3
束缚态与离散谱}\label{ux675fux7f1aux6001ux4e0eux79bbux6563ux8c31}}

\hypertarget{ux675fux7f1aux6001ux80fdux7ea7}{%
\paragraph{{ }3.1 束缚态能级}\label{ux675fux7f1aux6001ux80fdux7ea7}}

  束缚能量本征态的能级是离散的

\hypertarget{ux6ce2ux51fdux6570ux6027ux8d28}{%
\paragraph{{ }3.2 波函数性质}\label{ux6ce2ux51fdux6570ux6027ux8d28}}

\begin{itemize}
\tightlist
\item
  在经典允许区,即\(U(x)<E\)时:波函数为震荡函数(\(sin\,kx,cos\,kx\)),\(\psi(x)\)总是向\(x\)轴弯曲
\item
  在经典禁区,即\(U(x)>E\)时:波函数为s单调函数(\(e^{\pm\beta x}\)),\(\psi(x)\)总是背离\(x\)轴弯曲
\end{itemize}

\hypertarget{ux57faux6001ux4e0eux6fc0ux53d1ux6001}{%
\paragraph{{ }3.3
基态与激发态}\label{ux57faux6001ux4e0eux6fc0ux53d1ux6001}}

\begin{itemize}
\tightlist
\item
  基态:除\(\pm\infty\)外,在\(x\)有限的区域内基态波函数无节点
\item
  激发态:随能级递增,波函数节点数一次增加一个
\end{itemize}

\hypertarget{ux516dux91cfux5b50ux96a7ux7a7fux6548ux5e94}{%
\subsection{六、量子隧穿效应}\label{ux516dux91cfux5b50ux96a7ux7a7fux6548ux5e94}}

\hypertarget{ux96a7ux7a7fux6548ux5e94}{%
\subsubsection{1 隧穿效应}\label{ux96a7ux7a7fux6548ux5e94}}

\hypertarget{ux52bfux80fdux51fdux6570-2}{%
\paragraph{{ }1.1 势能函数}\label{ux52bfux80fdux51fdux6570-2}}

\[U(x)=\begin{cases} 
        U_0, & 0<x<a\\
        0, & x<0,\,x>a
\end{cases}\]   考虑\(0<E<U_0\)时的情况

\hypertarget{ux5b9aux6001ux859bux5b9aux8c14ux65b9ux7a0b-4}{%
\paragraph{{ }1.2
定态薛定谔方程}\label{ux5b9aux6001ux859bux5b9aux8c14ux65b9ux7a0b-4}}

\[\begin{cases} 
        \frac{d^2\psi_1(x)}{dx^2}+\frac{2mE}{\hbar^2}\psi_1(x)=0 & x<0\\
        \frac{d^2\psi_2(x)}{dx^2}-\frac{2m(U_0-E)}{\hbar^2}\psi_2(x)=0, & 0<x<a\\
        \frac{d^2\psi_3(x)}{dx^2}+\frac{2mE}{\hbar^2}\psi_3(x)=0 & x>a
\end{cases}\]

\hypertarget{ux5217ux5199ux901aux89e3-2}{%
\paragraph{{ }1.3 列写通解}\label{ux5217ux5199ux901aux89e3-2}}

\[\begin{cases} 
        \psi_1(x)=e^{ikx}+Be^{-ikx} & x<0\\
        \psi_2(x)=Ce^{\kappa x}+De^{-\kappa x}, & 0<x<a\\
        \psi_3(x)=Se^{ikx} & x>a
\end{cases}\]
  其中\(k=\sqrt{\frac{2mE}{\hbar^2}},\,\kappa=\sqrt{\frac{2m(U_0-E)}{\hbar^2}}\)

\hypertarget{ux7269ux7406ux5206ux6790-1}{%
\paragraph{{ }1.4 物理分析}\label{ux7269ux7406ux5206ux6790-1}}

\begin{itemize}
\tightlist
\item
  透射波:从物理上分析,透射波\(\psi_3(x)\)中只可能存在向右传播的\(e^{ikx}\)波,因此不存在\(e^{-ikx}\)项
\item
  概率流密度:\(\vec{J}=\frac{i\hbar}{2m}(\Psi\nabla\Psi^* -\Psi^*\nabla\Psi)\),故有\\
   \(J_i=|1|^2\frac{\hbar k}{m}=v\)\\
   \(J_r=|B|^2\frac{\hbar k}{m}=|B|^2v\)\\
   \(J_t=|S|^2\frac{\hbar k}{m}=|S|^2v\)
\item
  透射概率:\(T=\frac{J_t}{J_i}=|S|^2\)
\item
  反射概率:\(R=\frac{J_r}{J_i}=|B|^2\)
\end{itemize}

\hypertarget{ux8fb9ux754cux6761ux4ef6-2}{%
\paragraph{{ }1.5 边界条件}\label{ux8fb9ux754cux6761ux4ef6-2}}

\begin{itemize}
\tightlist
\item
  利用边界条件有: \[\begin{cases} 
        \psi_1(0)=\psi_2(0)\\
        \psi_1'(0)=\psi_2'(0)\\
        \psi_2(a)=\psi_3(a)\\
        \psi_2'(a)=\psi_3'(a)
  \end{cases}\Rightarrow
  \begin{cases} 
        1+B=C+D\\
        ik(1-B)=\kappa(C-D)\\
        Ce^{\kappa a}+De^{-\kappa a}=Se^{ika}\\
        \kappa Ce^{\kappa a}-\kappa De^{-\kappa a}=ikSe^{ika}
  \end{cases}\]
\item
  解得:
  \[T=|S|^2=\frac{4k^2\kappa^2}{(k^2+\kappa^2)^2sh^2\kappa a+4k^2\kappa^2}\]\\
  \[R=|B|^2=\frac{(k^2+\kappa^2)^2sh^2\kappa a}{(k^2+\kappa^2)^2sh^2\kappa a+4k^2\kappa^2}\]
\end{itemize}

\hypertarget{ux8ba8ux8bba-4}{%
\paragraph{{ }1.6 讨论}\label{ux8ba8ux8bba-4}}

\begin{itemize}
\tightlist
\item
  概率非负:分析下式可知 \(0<T<1\) \[
  T=[1+\frac{(k^2+\kappa^2)^2}{4k^2\kappa^2}sh^2\kappa a]^{-1}
  =[1+\frac{1}{4\frac{E}{U_0}(1-\frac{E}{U_0})}sh^2\kappa a]^{-1}\]
\item
  概率守恒:\(R+T=1\)
\item
  近似公式:若满足条件\(\kappa a\gg1\),利用\(sh\,\kappa a\approx\frac{1}{2}e^{\kappa a}\gg1\)可得
  \[
  T\approx\frac{16k^2\kappa^2}{(k^2+\kappa^2)^2}e^{-2\kappa a}\approx T_0e^{-\frac{2a}{\hbar}\sqrt{2m(U_0-E)}}
  \]
  分析可知\(T\)敏感地依赖于势垒高度\(U_0\),宽度\(a\),粒子质量\(m\)和能量\(E\)
\item
  粒子隧穿一般形状势垒的透射概率:
  \[T\approx T_0e^{-\frac{2}{\hbar}\int_a^b\sqrt{2m(U(x)-E)}dx}\]
\end{itemize}

\hypertarget{ux5171ux632fux96a7ux7a7f}{%
\subsubsection{2 共振隧穿}\label{ux5171ux632fux96a7ux7a7f}}

  考虑\(E>U_0\)时的情况

\hypertarget{ux5b9aux6001ux859bux5b9aux8c14ux65b9ux7a0b-5}{%
\paragraph{{ }2.1
定态薛定谔方程}\label{ux5b9aux6001ux859bux5b9aux8c14ux65b9ux7a0b-5}}

\[\begin{cases} 
        \frac{d^2\psi_1(x)}{dx^2}+\frac{2mE}{\hbar^2}\psi_1(x)=0 & x<0\\
        \frac{d^2\psi_2(x)}{dx^2}+\frac{2m(E-U_0)}{\hbar^2}\psi_2(x)=0, & 0<x<a\\
        \frac{d^2\psi_3(x)}{dx^2}+\frac{2mE}{\hbar^2}\psi_3(x)=0 & x>a
\end{cases}\]

\hypertarget{ux5217ux5199ux901aux89e3-3}{%
\paragraph{{ }2.2 列写通解}\label{ux5217ux5199ux901aux89e3-3}}

\[\begin{cases} 
        \psi_1(x)=e^{ikx}+Be^{-ikx} & x<0\\
        \psi_2(x)=Ce^{ik'x}+De^{-ik'x}, & 0<x<a\\
        \psi_3(x)=Se^{ikx} & x>a
\end{cases}\]
  其中\(k=\sqrt{\frac{2mE}{\hbar^2}},\,k'=\sqrt{\frac{2m(E-U_0)}{\hbar^2}}\)

\hypertarget{ux7269ux7406ux5206ux6790-2}{%
\paragraph{{ }2.3 物理分析}\label{ux7269ux7406ux5206ux6790-2}}

\begin{itemize}
\tightlist
\item
  注意到在\(0<E<U_0\)时,\(\kappa=\sqrt{\frac{2m(U_0-E)}{\hbar^2}}\),故可将下关系式代入前表达式即可求得透射概率\(T\)
  \[
  \kappa=ik,\,sh(ik'a)=i\,sin(k'a)
  \]
\item
  透射概率: \[
  T=[1+\frac{1}{4}(\frac{k}{k'}-\frac{k'}{k})^2)sin^2k'a]^{-1}
  \]
  分析可知:当\(k'a=n\pi\)时,\(T=1\)
\end{itemize}

\hypertarget{ux65b9ux52bfux9631ux7684ux53cdux5c04ux900fux5c04ux4e0eux5171ux632f}{%
\subsubsection{3
方势阱的反射、透射与共振}\label{ux65b9ux52bfux9631ux7684ux53cdux5c04ux900fux5c04ux4e0eux5171ux632f}}

\hypertarget{ux7269ux7406ux5206ux6790-3}{%
\paragraph{{ }3.1 物理分析}\label{ux7269ux7406ux5206ux6790-3}}

\begin{itemize}
\tightlist
\item
  将\(U_0=-U_0,\,k’=\sqrt{\frac{2m(E+U_0)}{\hbar^2}}\)代入前表达式可得
  \[
  T=[1+\frac{1}{4}(\frac{k}{k'}-\frac{k'}{k})^2)sin^2k'a]^{-1}\]
  \[
  =[1+\frac{sin^2k'a}{4\frac{E}{U_0}(1+\frac{E}{U_0})}]^{-1}
  \]
\end{itemize}

\hypertarget{ux8ba8ux8bba-5}{%
\paragraph{{ }3.2 讨论}\label{ux8ba8ux8bba-5}}

\begin{itemize}
\tightlist
\item
  若\(U_0=0\),则\(T=1\)
\item
  若\(U_0\neq0\),则\(T<1,\,|R|^2\neq0\)
\item
  当\(k'a=n\pi,\,n=1,2,3,\ldots\)时,\(T=1\),称为共振透射,此时有共振能级
  \[
  E_n=-U_0+\frac{n^2\pi^2\hbar^2}{2ma^2},\,n=1,2,3,\ldots
  \]
\item
  当\(k'a=(n+\frac{1}{2})\pi,\,n=1,2,3,\ldots\)时,反射最强
\end{itemize}

\hypertarget{ux4e03ux4e00ux7ef4ux8c10ux632fux5b50}{%
\subsection{七、一维谐振子}\label{ux4e03ux4e00ux7ef4ux8c10ux632fux5b50}}

\hypertarget{ux52bfux80fdux51fdux6570-3}{%
\subsubsection{1 势能函数}\label{ux52bfux80fdux51fdux6570-3}}

\[
U(x)=\frac{1}{2}m\omega^2x^2
\]

\hypertarget{ux5b9aux6001ux859bux5b9aux8c14ux65b9ux7a0b-6}{%
\subsubsection{2
定态薛定谔方程}\label{ux5b9aux6001ux859bux5b9aux8c14ux65b9ux7a0b-6}}

\[
(-\frac{\hbar^2}{2m}\frac{d^2}{dx^2}+\frac{1}{2}m\omega^2x^2)\psi(x)=E\psi(x)
\]

\hypertarget{ux65b9ux7a0bux6c42ux89e3}{%
\subsubsection{3 方程求解}\label{ux65b9ux7a0bux6c42ux89e3}}

\hypertarget{ux65e0ux91cfux7eb2ux53d8ux6362}{%
\paragraph{{ }3.1 无量纲变换}\label{ux65e0ux91cfux7eb2ux53d8ux6362}}

  令\(\xi=\sqrt{\frac{m\omega}{\hbar}}x=\alpha x,\,\lambda=\frac{2E}{\hbar\omega}\),则原方程化为
\[
\frac{d^2\psi}{d\xi^2}+(\lambda-\xi^2)\psi(\xi)=0
\]

\hypertarget{ux6e10ux8fd1ux5206ux6790}{%
\paragraph{{ }3.2 渐近分析}\label{ux6e10ux8fd1ux5206ux6790}}

  当\(\xi\to\pm\infty\)时,有\(\frac{d^2\psi}{d\xi^2}\approx\xi^2\psi(\xi)\),从而\(\psi(\xi)\sim e^{-\frac{1}{2}\xi^2}\),故设\(\psi(\xi)=H(\xi)e^{-\frac{1}{2}\xi^2}\),代入原方程得Hermite方程
\[
\frac{d^2H}{d\xi^2}-2\xi\frac{dH}{d\xi}+(\lambda-1)H=0
\]

\hypertarget{ux7ea7ux6570ux89e3ux6cd5}{%
\paragraph{{ }3.3 级数解法}\label{ux7ea7ux6570ux89e3ux6cd5}}

\begin{itemize}
\tightlist
\item
  对\(H(\xi)\)做幂级数展开\(H(\xi)=\sum_{k=0}^\infty a_k\xi^k\),代入Hermite方程得到递推关系
  \[
  a_{k+2}=\frac{2k-(\lambda-1)}{(k+2)(k+1)}a_k,\,k=0,1,2,\ldots
  \]
\item
  取\(\lambda=2n+1,\,n=0,1,2,\ldots\),则\(a_{k+2}=\frac{2(k-n)}{(k+2)(k+1)}a_k\),解得
  \[
  H_n(\xi)=(-1)^ne^{\xi^2}\frac{d^n}{d\xi^n}e^{-\xi^2},\,n=0,1,2,\ldots
  \]
\item
  简单的Hermite多项式:\\
   \(H_0(\xi)=1\)\\
   \(H_1(\xi)=2\xi\)\\
   \(H_2(\xi)=4\xi^2-2\)
\end{itemize}

\hypertarget{ux80fdux7ea7ux548cux6ce2ux51fdux6570}{%
\subsubsection{4
能级和波函数}\label{ux80fdux7ea7ux548cux6ce2ux51fdux6570}}

\hypertarget{ux80fdux7ea7}{%
\paragraph{{ }4.1 能级}\label{ux80fdux7ea7}}

\[
E_n=(n+\frac{1}{2})\hbar\omega,\,n=0,1,2,\ldots
\]

\hypertarget{ux6ce2ux51fdux6570}{%
\paragraph{{ }4.2 波函数}\label{ux6ce2ux51fdux6570}}

\[
\psi_n(x)=A_nH_n(\xi)e^{-\frac{1}{2}\xi^2}=A_nH_n(\alpha x)e^{-\frac{1}{2}\alpha^2x^2},\,n=0,1,2,\ldots
\]
  其中\(\alpha=\sqrt{\frac{m\omega}{\hbar}}\)

\hypertarget{ux5f52ux4e00ux5316ux7cfbux6570}{%
\paragraph{{ }4.3 归一化系数}\label{ux5f52ux4e00ux5316ux7cfbux6570}}

\[
A_n=\sqrt{\frac{\alpha}{\sqrt{\pi}2^nn!}},\,n=0,1,2,\ldots
\]

\hypertarget{hermiteux591aux9879ux5f0f}{%
\paragraph{{ }4.4 Hermite多项式}\label{hermiteux591aux9879ux5f0f}}

\[
H_n(\xi)=(-1)^ne^{\xi^2}\frac{d^n}{d\xi^n}e^{-\xi^2},\,n=0,1,2,\ldots
\]

\hypertarget{ux6027ux8d28}{%
\paragraph{{ }4.5 性质}\label{ux6027ux8d28}}

\begin{itemize}
\tightlist
\item
  宇称:\(\psi_n(-x)=(-1)^n\psi_n(x)\)
\item
  正交性:\(\int_{-\infty}^{+\infty}\psi_m(x)\psi_n(x)dx=\delta_{mn}\)
\end{itemize}

\hypertarget{ux8ba8ux8bba-6}{%
\subsubsection{5 讨论}\label{ux8ba8ux8bba-6}}

\begin{itemize}
\tightlist
\item
  零点能:\(E_0=\frac{1}{2}\hbar\omega\)
\item
  能量量子化:\(E_{n+1}-E_n=\hbar\omega\),源于粒子德布罗意波的自身干涉
\item
  宇称:能量本征态的宇称奇偶相间,基态为偶宇称
\item
  节点:\(\psi_n(x)\)有n个节点
\item
  与经典谐振子的比较:n较小时,概率分布与经典谐振子完全不同;\(n\to\infty\)时,概率分布趋于经典概率分布,能量量子化趋于能量取连续值
\item
  对应原理:在大量子数极限下,量子论将渐近地趋于经典理论
\end{itemize}

\end{document}
