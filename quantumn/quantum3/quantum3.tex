\PassOptionsToPackage{unicode=true}{hyperref} % options for packages loaded elsewhere
\PassOptionsToPackage{hyphens}{url}
%
\documentclass[UTF8,twocolumn]{ctexart}
\usepackage{lmodern}
\usepackage{amssymb,amsmath}
\usepackage{ifxetex,ifluatex}
\usepackage{fixltx2e} % provides \textsubscript
\ifnum 0\ifxetex 1\fi\ifluatex 1\fi=0 % if pdftex
  \usepackage[T1]{fontenc}
  \usepackage[utf8]{inputenc}
  \usepackage{textcomp} % provides euro and other symbols
\else % if luatex or xelatex
  \usepackage{unicode-math}
  \defaultfontfeatures{Ligatures=TeX,Scale=MatchLowercase}
\fi
% use upquote if available, for straight quotes in verbatim environments
\IfFileExists{upquote.sty}{\usepackage{upquote}}{}
% use microtype if available
\IfFileExists{microtype.sty}{%
\usepackage[]{microtype}
\UseMicrotypeSet[protrusion]{basicmath} % disable protrusion for tt fonts
}{}
\IfFileExists{parskip.sty}{%
\usepackage{parskip}
}{% else
\setlength{\parindent}{0pt}
\setlength{\parskip}{6pt plus 2pt minus 1pt}
}
\usepackage{hyperref}
\hypersetup{
            pdfborder={0 0 0},
            breaklinks=true}
\urlstyle{same}  % don't use monospace font for urls
\setlength{\emergencystretch}{3em}  % prevent overfull lines
\providecommand{\tightlist}{%
  \setlength{\itemsep}{0pt}\setlength{\parskip}{0pt}}
\setcounter{secnumdepth}{0}
% Redefines (sub)paragraphs to behave more like sections
\ifx\paragraph\undefined\else
\let\oldparagraph\paragraph
\renewcommand{\paragraph}[1]{\oldparagraph{#1}\mbox{}}
\fi
\ifx\subparagraph\undefined\else
\let\oldsubparagraph\subparagraph
\renewcommand{\subparagraph}[1]{\oldsubparagraph{#1}\mbox{}}
\fi

% set default figure placement to htbp
\makeatletter
\def\fps@figure{htbp}
\makeatother


\date{}

\begin{document}

\hypertarget{ux7b2cux4e09ux7ae0-ux91cfux5b50ux529bux5b66ux4e2dux7684ux529bux5b66ux91cf}{%
\section{第三章{ }量子力学中的力学量}\label{ux7b2cux4e09ux7ae0-ux91cfux5b50ux529bux5b66ux4e2dux7684ux529bux5b66ux91cf}}

\hypertarget{ux4e00ux7b97ux7b26ux7684ux8fd0ux7b97ux53caux5bf9ux6613ux5173ux7cfb}{%
\subsection{一、算符的运算及对易关系}\label{ux4e00ux7b97ux7b26ux7684ux8fd0ux7b97ux53caux5bf9ux6613ux5173ux7cfb}}

\hypertarget{ux7b97ux7b26ux7684ux6784ux6210}{%
\subsubsection{1 算符的构成}\label{ux7b97ux7b26ux7684ux6784ux6210}}

\hypertarget{ux6982ux5ff5}{%
\paragraph{ 1.1 概念}\label{ux6982ux5ff5}}

  算符是作用于波函数后将其变成另一个函数的运算符号,代表力学量\(F\)的算符将记做\(\hat{F}\)

\hypertarget{ux57faux672cux5047ux5b9a}{%
\paragraph{ 1.2 基本假定}\label{ux57faux672cux5047ux5b9a}}

  量子力学中任一可观测力学量\(F\)都可以用线性厄米算符\(\hat{F}\)来表示

\hypertarget{ux6784ux6210}{%
\paragraph{ 1.3 构成}\label{ux6784ux6210}}

\begin{itemize}
\tightlist
\item
  坐标算符:\(\hat{\vec{r}}=\vec{r}\,(\hat{x}=x,\hat{y}=y,\hat{z}=z)\)
\item
  动量算符:\(\hat{\vec{p}}=-i\hbar\nabla\),其中
  \[\hat{p_x}=-i\hbar\frac{\partial}{\partial x},\hat{p_y}=-i\hbar\frac{\partial}{\partial y},\hat{p_z}=-i\hbar\frac{\partial}{\partial z}\]
\item
  一般力学量算符:\(F=f(\vec{r},\vec{p})\)
  \[\Rightarrow\hat{F}=f(\hat{\vec{r}},\hat{\vec{p}})=f(\hat{\vec{r}},-i\hbar\nabla)\]
\end{itemize}

\hypertarget{ux5e38ux89c1ux7b97ux7b26}{%
\paragraph{ 1.4 常见算符}\label{ux5e38ux89c1ux7b97ux7b26}}

\begin{itemize}
\tightlist
\item
  轨道角动量算符:\(\hat{\vec{L}}=\hat{\vec{r}}\times\hat{\vec{p}}=\vec{r}\times(i\hbar\nabla)=\vec{i}\hat{L}_x+\vec{j}\hat{L}_y+\vec{k}\hat{L}_z\)
  \[\begin{cases} 
    \hat{L}_x=y\hat{p}_z-z\hat{p}_y\\
    \hat{L}_y=z\hat{p}_x-x\hat{p}_z\\
    \hat{L}_z=x\hat{p}_y-y\hat{p}_x
  \end{cases}\]
\item
  角动量平方算符:\(\hat{L}^2=\hat{L}_x^2+\hat{L}_y^2+\hat{L}_z^2\)
\item
  非相对论动能算符:\(\hat{T}=\frac{\hat{\vec{p}}^2}{2m}=-\frac{\hbar^2}{2m}\nabla^2\)
\item
  势能算符:\(\hat{U}=\hat{U}(\hat{\vec{r}})\)
\item
  能量算符:\(\hat{E}=i\hbar\frac{\partial}{\partial t}\)
\item
  哈密顿算符:\(\hat{H}=\hat{T}+\hat{V}\)
\end{itemize}

\hypertarget{ux7b97ux7b26ux7684ux8fd0ux7b97ux89c4ux5219}{%
\subsubsection{2
算符的运算规则}\label{ux7b97ux7b26ux7684ux8fd0ux7b97ux89c4ux5219}}

 设\(\Psi,\Phi\)为任意波函数,\(C_1,C_2\)为任意复常数

\hypertarget{ux7ebfux6027ux7b97ux7b26}{%
\paragraph{ 2.1 线性算符}\label{ux7ebfux6027ux7b97ux7b26}}

\begin{itemize}
\tightlist
\item
  定义:\(\hat{A}(C_1\Psi+C_2\Phi)=C_1\hat{A}\Psi+C_2\hat{A}\Phi\)
\item
  注:并非所有算符都是线性算符,但刻画可观测量的算符都是线性算符
\end{itemize}

\hypertarget{ux7b97ux7b26ux76f8ux7b49}{%
\paragraph{ 2.2 算符相等}\label{ux7b97ux7b26ux76f8ux7b49}}

\begin{center}
\(\hat{A}\Psi=\hat{B}\Psi\Rightarrow\hat{A}=\hat{B}\)
\end{center}

\hypertarget{ux5355ux4f4dux7b97ux7b26}{%
\paragraph{ 2.3 单位算符}\label{ux5355ux4f4dux7b97ux7b26}}

\begin{center}
\(\hat{I}\Psi=\Psi\)
\end{center}

\hypertarget{ux7b97ux7b26ux4e4bux548c}{%
\paragraph{ 2.4 算符之和}\label{ux7b97ux7b26ux4e4bux548c}}

\begin{itemize}
\tightlist
\item
  定义:\((\hat{A}+\hat{B})\Psi=\hat{A}\Psi+\hat{B}\Psi\)
\item
  交换律:\(\hat{A}+\hat{B}=\hat{B}+\hat{A}\)
\item
  结合律:\(\hat{A}+(\hat{B}+\hat{C})=(\hat{A}+\hat{B})+\hat{C}\)
\item
  注:两个线性算符之和仍为线性算
\end{itemize}

\hypertarget{ux7b97ux7b26ux4e4bux79ef}{%
\paragraph{ 2.5 算符之积}\label{ux7b97ux7b26ux4e4bux79ef}}

\begin{itemize}
\tightlist
\item
  定义:\((\hat{A}\hat{B})\Psi=\hat{A}(\hat{B}\Psi)\)
\item
  幂运算:\(\hat{A}^n=\hat{A}\hat{A}\cdots\hat{A}\)
\item
  注:算符之积一般不满足交换律,即
    \[\hat{A}\hat{B}\neq\hat{B}\hat{A}\]
\end{itemize}

\hypertarget{ux9006ux7b97ux7b26}{%
\paragraph{ 2.6 逆算符}\label{ux9006ux7b97ux7b26}}

\begin{itemize}
\tightlist
\item
  定义:若由\(\hat{A}\Psi=\Phi\)能唯一地解出\(\Psi\),则可定义\(\hat{A}\)的逆算符\(\hat{A}^{-1}\),满足\(\Psi=\hat{A}^{-1}\Phi\)
\item
  性质1:\(\hat{A}\hat{A}^{-1}=\hat{A}^{-1}\hat{A}=\hat{I}\)
\item
  性质2:\((\hat{A}\hat{B})^{-1}=\hat{B}^{-1}\hat{A}^{-1}\)
\item
  注:并非所有算符都有逆算符,如投影算符就不存在逆算符
\end{itemize}

\hypertarget{ux7b97ux7b26ux7684ux590dux5171ux8f6d}{%
\paragraph{ 2.7
算符的复共轭}\label{ux7b97ux7b26ux7684ux590dux5171ux8f6d}}

\begin{itemize}
\tightlist
\item
  定义:将\(\hat{A}\)表达式中的所有量换为其复共轭,即为\(\hat{A}^*\)
\item
  注:算符的复共轭\(\hat{A}^*\)与表象有关
\end{itemize}

\hypertarget{ux7b97ux7b26ux7684ux5384ux7c73ux5171ux8f6d}{%
\paragraph{ 2.8
算符的厄米共轭}\label{ux7b97ux7b26ux7684ux5384ux7c73ux5171ux8f6d}}

\begin{itemize}
\tightlist
\item
  定义:\(\hat{A}^+\),满足
  $$(\Psi(x),\hat{A}^+\Phi(x))=(\hat{A}\Psi(x),\Phi(x))$$
\item
  性质1:\((\hat{A}^+)^+=\hat{A}\)
\item
  性质2:\((\hat{A}+\hat{B})^+=\hat{A}^++\hat{B}^+\)
\item
  性质3:\((\hat{A}\hat{B})^+=\hat{B}^+\hat{A}^+\)
\end{itemize}

\hypertarget{ux5384ux7c73ux7b97ux7b26}{%
\paragraph{ 2.9 厄米算符}\label{ux5384ux7c73ux7b97ux7b26}}

\begin{itemize}
\tightlist
\item
  定义:若\(\hat{F}^+=\hat{F}\),则称\(\hat{F}\)为厄米算符
\item
  性质1:\((\hat{F}\Psi,\Phi)=(\Psi,\hat{F}\Phi)\)
\item
  性质2:厄米算符的本征值都是实数
\item
  性质3:厄米算符之和仍为厄米算符
\item
  性质4:厄米算符之积不一定为厄米算符
\end{itemize}

\hypertarget{ux72b6ux6001ux7684ux529bux5b66ux91cfux671fux671b}{%
\paragraph{ 2.10
状态的力学量期望}\label{ux72b6ux6001ux7684ux529bux5b66ux91cfux671fux671b}}

\begin{itemize}
\tightlist
\item
  定义:若体系的波函数为\(\Psi\),则其力学量\(A\)的期望为\(\overline{A}=(\Psi,\hat{A}\Psi)\)
\item
  性质1:体系的任何状态下,其厄米算符对应力学量的期望为实数
\item
  性质2:若体系的任何状态下,某力学量的期望均为实数,则该力学量对应的算符为厄米算符
\item
  性质3:若\(\hat{A}\)为厄米算符,则\(\overline{A^2}=(\Psi,\hat{A}^2\Psi)=(\hat{A}\Psi,\hat{A}\Psi)\geq0\)
\end{itemize}

\hypertarget{ux5e7aux6b63ux7b97ux7b26}{%
\paragraph{ 2.11 幺正算符}\label{ux5e7aux6b63ux7b97ux7b26}}

\begin{center}
$\hat{A}^+=\hat{A}^{-1}\Leftrightarrow\hat{A}\hat{A}^+=\hat{A}^+\hat{A}=\hat{I}$
\end{center}

\hypertarget{ux7b97ux7b26ux7684ux51fdux6570}{%
\paragraph{ 2.12 算符的函数}\label{ux7b97ux7b26ux7684ux51fdux6570}}

  若给定函数\(F(x)=\sum_{n=0}^\infty\frac{F^{(n)}(0)}{n!}x^n\),则可定义算符\(\hat{A}\)的函数
\[
    F(\hat{A})=\sum_{n=0}^\infty\frac{F^{(n)}(0)}{n!}\hat{A}^n
\]

\hypertarget{ux7b97ux7b26ux7684ux672cux5f81ux65b9ux7a0bux53caux5176ux610fux4e49}{%
\subsubsection{3
算符的本征方程及其意义}\label{ux7b97ux7b26ux7684ux672cux5f81ux65b9ux7a0bux53caux5176ux610fux4e49}}

\hypertarget{ux7b97ux7b26ux7684ux672cux5f81ux65b9ux7a0b}{%
\paragraph{ 3.1
算符的本征方程}\label{ux7b97ux7b26ux7684ux672cux5f81ux65b9ux7a0b}}

\begin{center}
$\hat{F}\Psi_{\lambda}=\lambda\Psi_{\lambda}$
\end{center}

\hypertarget{ux6d4bux91cfux7684ux57faux672cux5047ux8bbe}{%
\paragraph{ 3.2
测量的基本假设}\label{ux6d4bux91cfux7684ux57faux672cux5047ux8bbe}}

\begin{itemize}
\tightlist
\item
  算符\(\hat{F}\)的本征值集\({\lambda}\)就是力学量\(F\)的测量值集
\item
  \(\hat{F}\)的本征函数\(\Psi_{\lambda}\)代表力学量\(F\)有确定值\(\lambda\)的状态
\end{itemize}

\hypertarget{ux7b97ux7b26ux51fdux6570ux7684ux672cux5f81ux503c}{%
\paragraph{ 3.3
算符函数的本征值}\label{ux7b97ux7b26ux51fdux6570ux7684ux672cux5f81ux503c}}

  若算符\(\hat{A}\)的本征值为\(\lambda\),则算符函数\(F(\hat{A})\)的本征值为\(F(\lambda)\)

\hypertarget{ux7b97ux7b26ux7684ux5bf9ux6613ux5173ux7cfb}{%
\subsubsection{4
算符的对易关系}\label{ux7b97ux7b26ux7684ux5bf9ux6613ux5173ux7cfb}}

\hypertarget{ux5b9aux4e49}{%
\paragraph{ 4.1 定义}\label{ux5b9aux4e49}}

  记\([\hat{A},\hat{B}]=\hat{A}\hat{B}-\hat{B}\hat{A}\),若\([\hat{A},\hat{B}]=0\),即\(\hat{A}\hat{B}=\hat{B}\hat{A}\),则称\(\hat{A}\)和\(\hat{B}\)对易

\hypertarget{ux8fd0ux7b97ux89c4ux5219}{%
\paragraph{ 4.2 运算规则}\label{ux8fd0ux7b97ux89c4ux5219}}

\begin{itemize}
\tightlist
\item
  \([\hat{A},\hat{B}]=-[\hat{B},\hat{A}]\)
\item
  \([\hat{A},\hat{B}+\hat{C}]=[\hat{A},\hat{B}]+[\hat{A},\hat{C}]\)
\item
  \([\hat{A},\hat{B}\hat{C}]=\hat{B}[\hat{A},\hat{C}]+[\hat{A},\hat{B}]\hat{C}\)
\item
  \([\hat{A}\hat{B},\hat{C}]=\hat{A}[\hat{B},\hat{C}]+[\hat{A},\hat{C}]\hat{B}\)
\item
  \([\hat{A},[\hat{B},\hat{C}]]+[\hat{B},[\hat{C},\hat{A}]]+[\hat{C},[\hat{A},\hat{B}]]=0\)
\end{itemize}

\hypertarget{ux5750ux6807ux52a8ux91cfux7684ux5bf9ux6613ux5173ux7cfb}{%
\paragraph{ 4.3
坐标、动量的对易关系}\label{ux5750ux6807ux52a8ux91cfux7684ux5bf9ux6613ux5173ux7cfb}}

\begin{itemize}
\tightlist
\item
  \([x,y]=[y,z]=[z,x]=0\)
\item
  \([\hat{p}_x,\hat{p}_y]=[\hat{p}_y,\hat{p}_z]=[\hat{p}_z,\hat{p}_x]=0\)
\item
  \([x,\hat{p}_x]=[y,\hat{p}_y]=[z,\hat{p}_z]=i\hbar\)
\item
  \([x,\hat{p}_y]=[x,\hat{p}_z]=\cdots=0\)
\end{itemize}

\hypertarget{ux89d2ux52a8ux91cfux7684ux5bf9ux6613ux5173ux7cfb}{%
\paragraph{ 4.4
角动量的对易关系}\label{ux89d2ux52a8ux91cfux7684ux5bf9ux6613ux5173ux7cfb}}

\begin{itemize}
\item
  角动量算符:如果一个矢量算符的三个分量满足下述对易关系,则这个算符为角动量算符
  \[\begin{cases} 
    [\hat{L}_x,\hat{L}_y]=i\hbar\hat{L}_z\\
    [\hat{L}_y,\hat{L}_z]=i\hbar\hat{L}_x\\
    [\hat{L}_z,\hat{L}_x]=i\hbar\hat{L}_y
  \end{cases}\]
\item
  性质:角动量算符的三个分量都和角动量的平方对易 \[
    [\hat{L}_x,\hat{L}^2]=[\hat{L}_y,\hat{L}^2]=[\hat{L}_z,\hat{L}^2]=0
  \]
\item
  轨道角动量算符:\(\hat{\vec{L}}=\vec{r}\times\hat{\vec{p}}=\vec{i}\hat{L}_x+\vec{j}\hat{L}_y+\vec{k}\hat{L}_z\)
  \[\begin{cases} 
    \hat{L}_x=y\hat{p}_z-z\hat{p}_y\\
    \hat{L}_y=z\hat{p}_x-x\hat{p}_z\\
    \hat{L}_z=x\hat{p}_y-y\hat{p}_x
  \end{cases}\]
\item
  轨道角动量和坐标之间的对易关系
  \begin{center}
  $[\hat{l}_{\alpha},\hat{x}_{\beta}]=\varepsilon_{\alpha\beta\gamma}i\hbar\hat{x}_{\gamma}$
  \end{center}
\begin{equation*}
  \begin{array}{lll}
    {[\hat{L}_x,\hat{x}]}=0 & [\hat{L}_x,\hat{y}]=i\hbar\hat{z} & [\hat{L}_x,\hat{z}]=-i\hbar\hat{y} \\
    {[\hat{L}_y,\hat{x}]}=-i\hbar\hat{z} & [\hat{L}_y,\hat{y}]=0 & [\hat{L}_y,\hat{z}]=i\hbar\hat{x} \\
    {[\hat{L}_z,\hat{x}]}=i\hbar\hat{y} & [\hat{L}_z,\hat{y}]=-i\hbar\hat{x} & [\hat{L}_z,\hat{z}]=0 \\
  \end{array}
\end{equation*}
\item
  轨道角动量和动量之间的对易关系
  \begin{center}
  $[\hat{l}_{\alpha},\hat{p}_{\beta}]=\varepsilon_{\alpha\beta\gamma}i\hbar\hat{p}_{\gamma}$
  \end{center}
  \begin{equation*}
  \begin{array}{lll}
    {[\hat{L}_x,\hat{p}_x]}=0 & [\hat{L}_x,\hat{p}_y]=i\hbar\hat{p}_z & [\hat{L}_x,\hat{p}_z]=-i\hbar\hat{p}_y\\
    {[\hat{L}_y,\hat{p}_x]}=-i\hbar\hat{p}_z & [\hat{L}_y,\hat{p}_y]=0 & [\hat{L}_y,\hat{p}_z]=i\hbar\hat{p}_x\\
    {[\hat{L}_z,\hat{p}_x]}=i\hbar\hat{p}_y & [\hat{L}_z,\hat{p}_y]=-i\hbar\hat{p}_x & [\hat{L}_z,\hat{p}_z]=0
  \end{array}
  \end{equation*}
\end{itemize}

\hypertarget{ux5171ux540cux672cux5f81ux51fdux6570}{%
\subsubsection{5
共同本征函数}\label{ux5171ux540cux672cux5f81ux51fdux6570}}

\hypertarget{ux5b9aux7406}{%
\paragraph{ 5.1 定理}\label{ux5b9aux7406}}

  若算符\(\hat{F}\)和\(\hat{G}\)有一组共同本征函数\(\phi_n\),且\({\phi_n}\)组成完全系,则算符\(\hat{F}\)和\(\hat{G}\)对易。

\hypertarget{ux9006ux5b9aux7406}{%
\paragraph{ 5.2 逆定理}\label{ux9006ux5b9aux7406}}

\begin{itemize}
\tightlist
\item
  非简并:若算符\(\hat{F}\)和\(\hat{G}\)均非简并,且\([\hat{F},\hat{G}]=0\),则\(\hat{F}\)和\(\hat{G}\)有一组共同本征函数\(\phi_n\),且\({\phi_n}\)组成完全系。
\item
  简并:若算符\(\hat{F}\)和\(\hat{G}\)存在简并,且\([\hat{F},\hat{G}]=0\),则存在\(\phi\),使得\(\hat{F}\phi=\lambda\phi\)和\(\hat{G}\phi=\mu\phi\)同时成立。
\end{itemize}

\hypertarget{ux63a8ux5e7f}{%
\paragraph{ 5.3 推广}\label{ux63a8ux5e7f}}

  如果一组算符有共同的本征函数,且这些本征函数组成完全系,则这组算符中的任意两个算符均对易。

\hypertarget{ux5b9eux4f8b}{%
\paragraph{ 5.4 实例}\label{ux5b9eux4f8b}}

\begin{itemize}
\tightlist
\item
  \(\{\hat{x},\hat{y},\hat{z}\}\)的共同本征函数为
  $$\Psi_{x_0,y_0,z_0}(x,y,z)=\delta(x-x_0)\delta(y-y_0)\delta(z-z_0)$$
\item
  \(\{\hat{p}_x,\hat{p}_y,\hat{p}_z\}\)的共同本征函数为
  $$\Psi_{p_x,p_y,p_z}(x,y,z)=\frac{1}{(2\pi\hbar)^{3/2}}e^{\frac{i}{\hbar}(p_xx+p_yy+p_zz)}$$
\item
  \(\{\hat{L}^2,\hat{L}_z\}\)的共同本征函数为球谐函数\(Y_{lm}(\theta,\varphi)\)
\item
  \(\{\hat{H},\hat{L}^2,\hat{L}_z\}\)的共同本征函数为\(\varPsi_{nlm}(r,\theta,\varphi)\)
\end{itemize}

\hypertarget{ux4e8cux7b97ux7b26ux4e0eux529bux5b66ux91cfux4e4bux95f4ux7684ux5173ux7cfb}{%
\subsection{二、算符与力学量之间的关系}\label{ux4e8cux7b97ux7b26ux4e0eux529bux5b66ux91cfux4e4bux95f4ux7684ux5173ux7cfb}}

\hypertarget{ux5384ux7c73ux7b97ux7b26-1}{%
\subsubsection{1 厄米算符}\label{ux5384ux7c73ux7b97ux7b26-1}}

\hypertarget{ux672cux5f81ux503c}{%
\paragraph{ 1.1 本征值}\label{ux672cux5f81ux503c}}

\begin{itemize}
\tightlist
\item
  分立谱:\(\hat{F}\Phi_n(x)=a_n\Phi_n(x),\,n=1,2,3,\cdots\)
\item
  连续谱:\(\hat{F}\Phi_{\lambda}(x)=\lambda\Phi_{\lambda}(x)\)
\end{itemize}

\hypertarget{ux6027ux8d28}{%
\paragraph{ 1.2 性质}\label{ux6027ux8d28}}

  厄米算符的本征值为实数。

\hypertarget{ux529bux5b66ux91cfux5b8cux5168ux96c6}{%
\subsubsection{2
力学量完全集}\label{ux529bux5b66ux91cfux5b8cux5168ux96c6}}

\hypertarget{ux5b9aux4e49-1}{%
\paragraph{ 2.1 定义}\label{ux5b9aux4e49-1}}

\begin{itemize}
\tightlist
\item
  定义:两两对易的,能对体系状态进行不简并地分类标记的,最少数目的一组力学量算符。
\item
  力学量完全集:\(\{\hat{F}_1,\hat{F}_2,\cdots,\hat{F}_n\}\)
\item
  共同本征函数系:\(\{\Phi_{a_1,a_2,\cdots,a_n}\}\)
\item
  本征值:\(\{a_1,a_2,\cdots,a_n\}\),一组本征值可完全确定体系的一个可能状态。
\end{itemize}

\hypertarget{ux5750ux6807}{%
\paragraph{ 2.2 坐标}\label{ux5750ux6807}}

\begin{itemize}
\tightlist
\item
  力学量完全集:\(\{\hat{x},\hat{y},\hat{z}\}\)
\item
  共同本征函数系:
  $$\Psi_{x_0,y_0,z_0}(x,y,z)=\delta(x-x_0)\delta(y-y_0)\delta(z-z_0)$$
\item
  本征值:\(\{x_0,y_0,z_0\}\)
\end{itemize}

\hypertarget{ux52a8ux91cf}{%
\paragraph{ 2.3 动量}\label{ux52a8ux91cf}}

\begin{itemize}
\tightlist
\item
  力学量完全集:\(\{\hat{p}_x,\hat{p}_y,\hat{p}_z\}\)
\item
  共同本征函数系:
  $$\Psi_{p_x,p_y,p_z}(x,y,z)=\frac{1}{(2\pi\hbar)^{3/2}}e^{\frac{i}{\hbar}(p_xx+p_yy+p_zz)}$$
\item
  本征值:\(\{p_x,p_y,p_z\}\)
\end{itemize}

\hypertarget{ux8f6cux52a8}{%
\paragraph{ 2.4 转动}\label{ux8f6cux52a8}}

\begin{itemize}
\tightlist
\item
  力学量完全集:\(\{\hat{L}^2,\hat{L}_z\}\)
\item
  球坐标系下的角动量算符:
  \[\begin{cases}
  \hat{L}_z=-i\hbar\frac{\partial}{\partial\varphi}\\
  \hat{L}^2=-\hbar^2[\frac{1}{sin\,\theta}\frac{\partial}{\partial\theta}(sin\,\theta\frac{\partial}{\partial\theta})+\frac{1}{sin^2\theta}\frac{\partial^2}{\partial\varphi^2}]
  \end{cases}\]
\end{itemize}

\hypertarget{ux6b63ux4ea4ux6027ux5b9aux7406}{%
\subsubsection{3 正交性定理}\label{ux6b63ux4ea4ux6027ux5b9aux7406}}

\hypertarget{ux6b63ux4ea4}{%
\paragraph{ 3.1 正交}\label{ux6b63ux4ea4}}

\begin{center}
\(\int_{\infty}\Psi_1^*(\vec{r})\Psi_2(\vec{r})d\tau=0\)
\end{center}

\hypertarget{ux6b63ux4ea4ux6027ux5b9aux7406-1}{%
\paragraph{ 3.2 正交性定理}\label{ux6b63ux4ea4ux6027ux5b9aux7406-1}}

\begin{itemize}
\tightlist
\item
  定理:厄米算符不同本征值的本征波函数彼此正交,相同本征值存在一组完备的正交本征波函数。
\item
  分立谱: 
  \begin{center}
  $(\Phi_l,\Phi_{l'})=\delta_{l,l'}=
  \begin{cases}
    0, & l\neq l'\\
    1, & l=l'
  \end{cases}$
  \end{center}
\item
  连续谱: 
  $$(\Phi_{\lambda},\Phi_{\lambda'})=\delta(\lambda-\lambda')$$
\end{itemize}

\hypertarget{ux672cux5f81ux51fdux6570ux7cfbux7684ux5b8cux5907ux6027}{%
\subsubsection{4
本征函数系的完备性}\label{ux672cux5f81ux51fdux6570ux7cfbux7684ux5b8cux5907ux6027}}

\hypertarget{ux5b9aux4e49-2}{%
\paragraph{ 4.1 定义}\label{ux5b9aux4e49-2}}

  若一个函数系完备,则任何一个满足适当边界条件和连续性要求的波函数均可用这个函数系作展开。

\hypertarget{ux5c01ux95edux5173ux7cfb}{%
\paragraph{ 4.2 封闭关系}\label{ux5c01ux95edux5173ux7cfb}}

\begin{itemize}
\tightlist
\item
  分立谱:\(\sum_n\Phi_n^*(x')\Phi_n(x)=\delta(x'-x)\)
\item
  连续谱:\(\int\Phi_{\lambda}^*(x')\Phi_{\lambda}(x)d\lambda=\delta(x'-x)\)
\item
  注意封闭关系和正交性定理的区别。
\end{itemize}

\hypertarget{ux529bux5b66ux91cfux7684ux6d4bux91cf}{%
\subsubsection{5
力学量的测量}\label{ux529bux5b66ux91cfux7684ux6d4bux91cf}}

\hypertarget{ux6982ux5ff5-1}{%
\paragraph{ 5.1 概念}\label{ux6982ux5ff5-1}}

\begin{itemize}
\tightlist
\item
  测量力学量F时所有可能出现的值,都是相应的线性厄米算符\(\hat{F}\)的本征值。
\item
  若体系处于\(\hat{F}\)的本征态\(\Psi_n\),则每次测量力学量A所得的结果是完全确定的,即\(\lambda_n\);
\item
  若体系处于\(\hat{F}\)的本征态叠加态,则单次测量结果可能值为本征态的本征值,出现的概率为对应本征态在该体系中的叠加系数的平方。
\end{itemize}

\hypertarget{ux6d4bux91cfux6982ux7387}{%
\paragraph{ 5.2 测量概率}\label{ux6d4bux91cfux6982ux7387}}

\begin{itemize}
\tightlist
\item
  分立谱:若\(\Psi(x)=\sum_nc_n\Phi_n(x)\),则单次测量值为\(\lambda_n\)的概率为\(|c_n|^2\),其中\(c_n=(\Psi,\Phi_n)\);
\item
  连续谱:若\(\Psi(x)=\int c_{\lambda}\Phi_{\lambda}(x)d\lambda\),则单次测量值出现在区间\([\lambda,\lambda+d\lambda]\)的概率为\(|c_{\lambda}|^2d\lambda\),其中\(c_{\lambda}=(\Psi,\Phi_{\lambda})\)。
\end{itemize}

\hypertarget{ux529bux5b66ux91cfux7684ux5e73ux5747ux503c}{%
\subsubsection{6
力学量的平均值}\label{ux529bux5b66ux91cfux7684ux5e73ux5747ux503c}}

\hypertarget{ux8ba1ux7b97ux65b9ux6cd5ux4e00}{%
\paragraph{ 6.1 计算方法一}\label{ux8ba1ux7b97ux65b9ux6cd5ux4e00}}

\begin{center}
\(\overline{F}=(\Psi(x),\hat{F}\Psi(x))=\int\Psi^*(x)\hat{F}\Psi(x)dx\)
\end{center}

\hypertarget{ux8ba1ux7b97ux65b9ux6cd5ux4e8c}{%
\paragraph{ 6.2 计算方法二}\label{ux8ba1ux7b97ux65b9ux6cd5ux4e8c}}

\begin{itemize}
\tightlist
\item
  分立谱:若\(\Psi(x)=\sum_nc_n\Phi_n(x)\),则
  \[\overline{F}=\sum_n|c_n|^2\lambda_n\]
\item
  连续谱:若\(\Psi(x)=\int c_{\lambda}\Phi_{\lambda}(x)d\lambda\),则
  \[\overline{F}=\int|c_{\lambda}|^2\lambda d\lambda\]
\end{itemize}

\hypertarget{ux4e09ux52a8ux91cfux7b97ux7b26ux548cux89d2ux52a8ux91cfux7b97ux7b26}{%
\subsection{三、动量算符和角动量算符}\label{ux4e09ux52a8ux91cfux7b97ux7b26ux548cux89d2ux52a8ux91cfux7b97ux7b26}}

\hypertarget{ux52a8ux91cfux7b97ux7b26}{%
\subsubsection{1 动量算符}\label{ux52a8ux91cfux7b97ux7b26}}

\hypertarget{ux7b97ux7b26}{%
\paragraph{ 1.1 算符}\label{ux7b97ux7b26}}

\begin{center}
\(\hat{\vec{p}}=-i\hbar\nabla\)
\end{center}

\hypertarget{ux672cux5f81ux503c-1}{%
\paragraph{ 1.2 本征值}\label{ux672cux5f81ux503c-1}}

\begin{center}
\(\vec{p}=(p_x,p_y,p_z)\)
\end{center}

\hypertarget{ux672cux5f81ux51fdux6570}{%
\paragraph{ 1.3 本征函数}\label{ux672cux5f81ux51fdux6570}}

\begin{center}
\(\Psi_{\vec{p}}(\vec{r})=\frac{1}{(2\pi\hbar)^{3/2}}e^{\frac{i}{\hbar}\vec{p}\cdot\vec{r}}\)
\end{center}

\hypertarget{zux65b9ux5411ux89d2ux52a8ux91cfux7b97ux7b26}{%
\subsubsection{2
z方向角动量算符}\label{zux65b9ux5411ux89d2ux52a8ux91cfux7b97ux7b26}}

\hypertarget{ux7b97ux7b26-1}{%
\paragraph{ 2.1 算符}\label{ux7b97ux7b26-1}}

\begin{center}
\(\hat{L}_z=-i\hbar\frac{\partial}{\partial\varphi}\)
\end{center}

\hypertarget{ux672cux5f81ux503c-2}{%
\paragraph{ 2.2 本征值}\label{ux672cux5f81ux503c-2}}

\begin{center}
\(L_z=m\hbar,\,m=0,\pm1,\pm2,\cdots\)
\end{center}

\hypertarget{ux672cux5f81ux51fdux6570-1}{%
\paragraph{ 2.3 本征函数}\label{ux672cux5f81ux51fdux6570-1}}

\begin{itemize}
\tightlist
\item
  本征函数:\(\Psi_m(\varphi)=\frac{1}{\sqrt{2\pi}}e^{im\varphi}\)
\item
  周期性边界条件:\(\Psi(\varphi+2\pi)=\Psi(\varphi)\)
\item
  归一化条件:\(\int_0^{2\pi}|\Psi(\varphi)|^2d\varphi=1\)
\end{itemize}

\hypertarget{ux89d2ux52a8ux91cfux5e73ux65b9ux7b97ux7b26}{%
\subsubsection{3
角动量平方算符}\label{ux89d2ux52a8ux91cfux5e73ux65b9ux7b97ux7b26}}

\hypertarget{ux7b97ux7b26-2}{%
\paragraph{ 3.1 算符}\label{ux7b97ux7b26-2}}

\begin{center}
\(\hat{L}^2=-\hbar^2[\frac{1}{\sin\,\theta}\frac{\partial}{\partial\theta}(sin\,\theta\frac{\partial}{\partial\theta})+\frac{1}{sin^2\theta}\frac{\partial^2}{\partial\varphi^2}]\)
\end{center}

\hypertarget{ux672cux5f81ux503c-3}{%
\paragraph{ 3.2 本征值}\label{ux672cux5f81ux503c-3}}

\begin{itemize}
\tightlist
\item
  \(L^2=l(l+1)\hbar^2,\,l=0,1,2,\cdots\)
\item
  \(L_z=m\hbar,\,m=0,\pm1,\cdots,\pm l\)
\end{itemize}

\hypertarget{ux672cux5f81ux51fdux6570-2}{%
\paragraph{ 3.3 本征函数}\label{ux672cux5f81ux51fdux6570-2}}

\begin{itemize}
\tightlist
\item
  球谐函数:\(Y_{lm}(\theta,\varphi)=N_{lm}P_l^m(cos\,\theta)e^{im\varphi}\)
\item
  正交性:\(\int Y_{l'm'}^*(\theta,\varphi)Y_{lm}(\theta,\varphi)d\Omega=\delta_{l'l}\delta_{m'm}\)
\item
  奇偶宇称:\(Y_{lm}^*(\theta,\varphi)=(-1)^mY_{l,-m}^*(\theta,\varphi)\)
\item
  实例: 
  \begin{center}
  $\begin{cases}
  Y_{00}=\sqrt{\frac{1}{4\pi}}\\
  Y_{10}=\sqrt{\frac{3}{4\pi}}cos\,\theta\\
  Y_{1,\pm1}=\mp\sqrt{\frac{3}{8\pi}}sin\,\theta e^{\pm i\phi}
  \end{cases}$
  \end{center}
  
\end{itemize}

\hypertarget{ux8bf4ux660e}{%
\paragraph{ 3.4 说明}\label{ux8bf4ux660e}}

\begin{itemize}
\tightlist
\item
  \(Y_{lm}(\theta,\varphi)\)是\(\hat{L}^2\)和\(\hat{L}_z\)的共同本征函数
\item
  $l$称为轨道量子数,m称为磁量子数;$l$为0, 1,2,3,\ldots{}的状态分别称为s,p,d,f,\ldots{}态
\item
  \(\hat{L}^2\)的本征值\(l(l+1)\hbar^2\)是(2l+1)度简并的
\item
  算符集合\(\{\hat{L}^2,\hat{L}_z\}\)是描述转动的力学量完全集,用量子数\{l,m\}可以完全确定转动态
\end{itemize}

\hypertarget{ux56dbux5b88ux6052ux91cf}{%
\subsection{四、守恒量}\label{ux56dbux5b88ux6052ux91cf}}

\hypertarget{ux5b88ux6052ux91cf}{%
\subsubsection{1 守恒量}\label{ux5b88ux6052ux91cf}}

\hypertarget{ux5b9aux4e49-3}{%
\paragraph{ 1.1 定义}\label{ux5b9aux4e49-3}}

  在体系的任何状态下,力学量F的平均值与取值概率分布都不随时间变化,则力学量F为体系的守恒量。

\hypertarget{ux5224ux5b9a}{%
\paragraph{ 1.2 判定}\label{ux5224ux5b9a}}

  若力学量算法\(\hat{A}\)不含时,且\([\hat{A},\hat{H}]=0\),则A为守恒量,即
\[\frac{d\overline{A}}{dt}=\frac{\overline{\partial\hat{A}}}{\partial t}+\frac{1}{i\hbar}\overline{[\hat{A},\hat{H}]}=0\]

\hypertarget{ux4e0eux7ecfux5178ux5b88ux6052ux91cfux7684ux533aux522b}{%
\paragraph{ 1.3
与经典守恒量的区别}\label{ux4e0eux7ecfux5178ux5b88ux6052ux91cfux7684ux533aux522b}}

\begin{itemize}
\tightlist
\item
  量子体系的守恒量不一定取确定值,而是有确定的期望值和概率分布;
\item
  量子体系的守恒量不一定都可以同时取确定值。
\end{itemize}

\hypertarget{ux4e0eux5b9aux6001ux7684ux533aux522b}{%
\paragraph{ 1.4
与定态的区别}\label{ux4e0eux5b9aux6001ux7684ux533aux522b}}

\begin{itemize}
\tightlist
\item
  定态是体系的状态,而守恒量是力学量;
\item
  在定态上,任意力学量的平均值和取值概率分布均不随时间变化;
\item
  在任意态上,守恒量的平均值和取值概率分布均不随时间变化。
\end{itemize}

\hypertarget{ux5b9eux4f8b-1}{%
\paragraph{ 1.5 实例}\label{ux5b9eux4f8b-1}}

\begin{itemize}
\tightlist
\item
  \(\hat{H}\)为任意状态的守恒量;
\item
  自由粒子的守恒量:\(\hat{x},\,\hat{\vec{p}},\,\hat{T},\,\hat{P},\,\hat{L}_z,\,\hat{L}^2,\,\hat{H}\);
\item
  中心力场中粒子的守恒量:\(\hat{H},\,\hat{L}^2,\,\hat{L}_z\)
\end{itemize}

\hypertarget{ux5b88ux6052ux91cfux4e0eux80fdux7ea7ux7b80ux5e76}{%
\subsubsection{2
守恒量与能级简并}\label{ux5b88ux6052ux91cfux4e0eux80fdux7ea7ux7b80ux5e76}}

\hypertarget{ux5b9aux7406-1}{%
\paragraph{ 2.1 定理}\label{ux5b9aux7406-1}}

\begin{itemize}
\tightlist
\item
  定理:如果体系有两个不对易的守恒量,则体系的能级一般是简并的。
\item
  推论:若体系有一个守恒量\(\hat{F}\),且体系的某条能级不简并,即对应于某能量本征值E只有一个本整天\(\Psi_E\),则\(\Psi_E\)必为\(\hat{F}\)的本征态。
\end{itemize}

\hypertarget{ux610fux4e49}{%
\paragraph{ 2.2 意义}\label{ux610fux4e49}}

  当能级出现简并是,可以根据对体系对称性的分析找出其守恒量;通过找到一组包含\(\hat{H}\)的对易守恒量完全集及其共同本征态,就可以把能级的各简并态标记清楚。

\hypertarget{ux4f4dux529bux5b9aux7406}{%
\subsubsection{3 位力定理}\label{ux4f4dux529bux5b9aux7406}}

\begin{itemize}
\tightlist
\item
  意义:定态体系力学量平均值随时间的变化。
\item
  定理:设粒子处在势场\(V{\vec{r}}\)中,即Hamilton量为\(H=\frac{p^2}{2m}+V(\vec{r})\),则粒子的动量算符在定态上的平均值为
  \[\overline{T}=\frac{1}{2}\overline{\vec{r}\cdot\nabla V}\]
\item
  推论:若势能为坐标的齐次函数\(V=\alpha x^n+\beta y^n+\gamma z^n\)时,有
  \[\overline{T}=\frac{n}{2}\overline{V}\]
\end{itemize}

\hypertarget{ux4e94ux4e0dux786eux5b9aux5ea6ux5173ux7cfb}{%
\subsection{五、不确定度关系}\label{ux4e94ux4e0dux786eux5b9aux5ea6ux5173ux7cfb}}

\hypertarget{ux4e0dux786eux5b9aux5ea6}{%
\subsubsection{1 不确定度}\label{ux4e0dux786eux5b9aux5ea6}}

\hypertarget{ux4e0dux786eux5b9aux5ea6-1}{%
\paragraph{ 1.1 不确定度}\label{ux4e0dux786eux5b9aux5ea6-1}}

\begin{itemize}
\tightlist
\item
  偏差算符:\(\Delta\hat{A}=\hat{A}-\overline{A}\)
\item
  不确定度:\(\Delta A=\sqrt{\overline{(\Delta\hat{A})^2}}=\sqrt{\overline{\hat{A}^2}-\overline{A}^2}\)
\end{itemize}

\hypertarget{ux4e0dux786eux5b9aux5ea6ux5173ux7cfb}{%
\paragraph{ 1.2
不确定度关系}\label{ux4e0dux786eux5b9aux5ea6ux5173ux7cfb}}

\begin{itemize}
\tightlist
\item
  定理:在任意态\(\Psi\)上任意两个力学量\(\hat{A}\)和\(\hat{B}\)的不确定度的乘积存在下限:
  \[\Delta A\cdot\Delta B\geq\frac{1}{2}|\overline{[\hat{A},\hat{B}]}|=\frac{1}{2}|\Psi,[\hat{A},\hat{B}]\Psi|\]
\item
  注1:当\([\hat{A},\hat{B}]\neq0\)时,除使得\((\Psi,[\hat{A},\hat{B}]\Psi)=0\)的特殊态\(\Psi\)外,在任何态上\(\hat{A}\)和\(\hat{B}\)都不能同时取确定值;
\item
  注2:当\([\hat{A},\hat{B}]=0\)时,\(\hat{A}\)和\(\hat{B}\)可以同时取确定值(但未必会取)。
\end{itemize}

\hypertarget{ux5750ux6807ux4e0eux52a8ux91cfux7684ux4e0dux786eux5b9aux5ea6ux5173ux7cfb}{%
\paragraph{ 1.3
坐标与动量的不确定度关系}\label{ux5750ux6807ux4e0eux52a8ux91cfux7684ux4e0dux786eux5b9aux5ea6ux5173ux7cfb}}

\begin{itemize}
\tightlist
\item
  公式:\(\Delta x\cdot\Delta p_x\leq\frac{\hbar}{2}\)
\item
  意义:粒子在客观上不能同时具有确定的坐标位置和确定的动量。
\end{itemize}

\hypertarget{ux65f6ux95f4ux4e0eux80fdux91cfux7684ux4e0dux786eux5b9aux5ea6ux5173ux7cfb}{%
\paragraph{ 1.4
时间与能量的不确定度关系}\label{ux65f6ux95f4ux4e0eux80fdux91cfux7684ux4e0dux786eux5b9aux5ea6ux5173ux7cfb}}

\begin{itemize}
\tightlist
\item
  公式:\(\Delta E\cdot\Delta t\leq\frac{\hbar}{2}\)
\item
  意义:若粒子在能量状态E只能停留\(\Delta t\)时间,则此时间段内能量有一弥散\(\Delta E\leq\hbar/(2\Delta t)\)
\end{itemize}

\hypertarget{ux4e0dux786eux5b9aux5ea6ux5173ux7cfbux5e94ux7528}{%
\subsubsection{2
不确定度关系应用}\label{ux4e0dux786eux5b9aux5ea6ux5173ux7cfbux5e94ux7528}}

\begin{itemize}
\tightlist
\item
  估计能级宽度与激发态寿命
\item
  证明原子中电子的运动不存在轨道
\item
  说明谐振子的零点能
\end{itemize}

\hypertarget{ux516dux4e2dux5fc3ux529bux573aux4e2dux7684ux7c92ux5b50ux8fd0ux52a8}{%
\subsection{六、中心力场中的粒子运动}\label{ux516dux4e2dux5fc3ux529bux573aux4e2dux7684ux7c92ux5b50ux8fd0ux52a8}}

\hypertarget{ux4e2dux5fc3ux529bux573a}{%
\subsubsection{1 中心力场}\label{ux4e2dux5fc3ux529bux573a}}

\begin{itemize}
\tightlist
\item
  特点:势能函数\(V(r)\)球对称,即\(V(r)\)与方向\(\theta,\phi\)无关
\item
  力学量完全集:\({\hat{H},\hat{L}^2},\hat{L}_z\)
\item
  本征值问题: \[\begin{cases}
  \hat{H}\Psi_{nlm}=E_{nl}\Psi_{nlm}\\
  \hat{L}^2\Psi_{nlm}=l(l+1)\hbar^2\Psi_{nlm}\\
  \hat{L}_z\Psi_{nlm}=m\hbar\Psi_{nlm}
  \end{cases}\]
\end{itemize}

\hypertarget{ux80fdux91cfux672cux5f81ux65b9ux7a0bux53caux5176ux6c42ux89e3}{%
\subsubsection{2
能量本征方程及其求解}\label{ux80fdux91cfux672cux5f81ux65b9ux7a0bux53caux5176ux6c42ux89e3}}

\begin{itemize}
\tightlist
\item
  Hamilton量:
  \[\hat{H}=\frac{\hat{p}^2}{2\mu}+V(r)=\frac{\hat{P}_r^2}{2\mu}+\frac{\hat{L}^2}{2\mu r^2}+V(r)\]
\item
  势函数:\(\Psi(r,\theta,\varphi)=R(r)Y_{lm}(\theta,\varphi)\)
\item
  宇称:本征态的宇称为\((-1)^l\)
\end{itemize}

\hypertarget{ux6c22ux539fux5b50ux7684ux672cux5f81ux51fdux6570}{%
\subsubsection{3
氢原子的本征函数}\label{ux6c22ux539fux5b50ux7684ux672cux5f81ux51fdux6570}}

\hypertarget{ux80fdux91cfux672cux5f81ux65b9ux7a0b}{%
\paragraph{ 3.1
能量本征方程}\label{ux80fdux91cfux672cux5f81ux65b9ux7a0b}}

\begin{itemize}
\tightlist
\item
  势函数:\(V(r)=-\frac{e^2}{r}\)
\item
  能量本征方程:
  \[[-\frac{\hbar^2}{2m_1}\nabla_1^2-\frac{\hbar^2}{2m_2}\nabla_2^2+V(r)]\Psi(\vec{r}_1,\vec{r}_2)=E_T\Psi(\vec{r}_1,\vec{r}_2)\]
\item
  质心坐标和相对坐标下的能量本征方程:
  \[[-\frac{\hbar^2}{2M}\nabla_R^2-\frac{\hbar^2}{2\mu}\nabla^2+V(r)]\Psi(\vec{R},\vec{r})=E_T\Psi(\vec{R},\vec{r})\]
\end{itemize}

\hypertarget{ux5f84ux5411ux65b9ux7a0b}{%
\paragraph{ 3.2 径向方程}\label{ux5f84ux5411ux65b9ux7a0b}}

\begin{itemize}
\tightlist
\item
  方程:\[\frac{d^2\chi_l(r)}{dr^2}+[2E+\frac{2}{r}-\frac{l(l+1)}{r^2}]\chi_l(r)=0\]
\item
  径向函数:\[R_{nl}(r)=N_{nl}exp(-\frac{\xi}{2})\xi^lF(-n+l+1,2l+2,\xi)\]
  其中\(\xi=\frac{2r}{na}\),F为合流超几何函数。
\item
  正交归一化条件:\[\int_0^\infty R_{nl}(r)R_{n'l'}(r)r^2dr=\delta_{n,n'}\delta_{l,l'}\]
\item
  实例: 
  \begin{equation*}
  \begin{array}{ll}
    n=1, & R_{10}(r)=\frac{2}{\sqrt{a^3}}e^{-\frac{r}{a}}\\
    n=2, & R_{20}(r)=\frac{1}{\sqrt{2a^3}}(1-\frac{r}{2a})e^{-\frac{r}{2a}}\\
    & R_{21}(r)=\frac{1}{2\sqrt{6a^3}}\frac{r}{a}e^{-\frac{r}{2a}}
  \end{array}
  \end{equation*}
\end{itemize}

\hypertarget{ux672cux5f81ux503cux4e0eux672cux5f81ux6ce2ux51fdux6570}{%
\paragraph{ 3.3
本征值与本征波函数}\label{ux672cux5f81ux503cux4e0eux672cux5f81ux6ce2ux51fdux6570}}

\begin{itemize}
\tightlist
\item
  本征值: \[\begin{cases}
  E_n=-\frac{e^2}{2an^2}=-\frac{13.6eV}{n^2}, & n=1,2,3,\ldots\\
  L^2=l(l+1)\hbar^2, & l=0,1,\ldots,n-1\\
  L_z=m\hbar, & m=0,\pm1,\ldots,\pm l
  \end{cases}\]
\item
  本征波函数:\(\Psi_{nlm}(r,\theta,\varphi)=R_{nl}(r)Y_{lm}(\theta,\phi)\)
\item
  归一化条件:
  $$\int\Psi_{nlm}^*(r,\theta,\varphi)\Psi_{n'l'm'}(r,\theta,\varphi)d\tau=\delta_{n,n'}\delta_{l,l'}\delta_{m,m'}$$
\end{itemize}

\hypertarget{ux6c22ux539fux5b50ux7684ux6027ux8d28}{%
\subsubsection{4
氢原子的性质}\label{ux6c22ux539fux5b50ux7684ux6027ux8d28}}

\hypertarget{ux80fdux7ea7ux7b80ux5e76ux5ea6}{%
\paragraph{ 4.1 能级简并度}\label{ux80fdux7ea7ux7b80ux5e76ux5ea6}}

\begin{itemize}
\tightlist
\item
  \(l=0,1,2,\ldots,n-1\)
\item
  \(m=0,\pm1,\pm2,\ldots,\pm l\)
\item
  \(f_n=\sum_{l=0}^{n-1}(2l+1)=n^2\)
\item
  若考虑自旋,则能级简并度为\(f_n=2n^2\)
\end{itemize}

\hypertarget{ux6982ux7387ux5bc6ux5ea6ux5206ux5e03}{%
\paragraph{ 4.2
概率密度分布}\label{ux6982ux7387ux5bc6ux5ea6ux5206ux5e03}}

\begin{itemize}
\tightlist
\item
  概率函数:
  \[W_{nlm}(r,\theta,\varphi)d\tau=|\Psi_{nlm}(r,\theta,\varphi)|^2r^2sin\theta drd\theta d\varphi\]
\item
  径向概率分布: \[\begin{aligned}
  W_{nl}(r)dr&=\int_0^{2\pi}\int_0^{\pi}|R_{nl}(r)Y_{lm}(\theta,\varphi)|^2r^2sin\theta drd\theta d\varphi\\
  &=R_{nl}^2(r)r^2dr
  \end{aligned}\]
\item
  角向概率分布: \[\begin{aligned}
  W_{lm}(\theta,\varphi)d\Omega&=\int_0^{+\infty}|R_{nl}(r)Y_{lm}(\theta,\varphi)|^2r^2sin\theta drd\theta d\varphi\\
  &=Y_{lm}^2(\theta,\varphi)sin\theta d\theta d\varphi
  \end{aligned}\]
\end{itemize}

\hypertarget{ux672cux5f81ux6001ux78c1ux77e9}{%
\paragraph{ 4.3 本征态磁矩}\label{ux672cux5f81ux6001ux78c1ux77e9}}

\begin{itemize}
\tightlist
\item
  轨道磁矩:\(\mu_z=-\mu_Bm\)
\item
  轨道磁矩算符:\(\hat{\vec{\mu}}_l=-\frac{\mu_B}{\hbar}\hat{\vec{L}}\)
\item
  轨道磁矩与外磁场的作用能:\[\hat{W}=-\hat{\vec{\mu}}_l\cdot\vec{B}=\frac{\mu_B}{\hbar}\hat{\vec{L}}\cdot\vec{B}\]
\end{itemize}

\end{document}
